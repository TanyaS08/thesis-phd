%%%%%%%%%%%%%%%%%%%%%%%%%%%%%%%%%%
%%%%%%%%% Préambule dans preamble.sty %%%%%%%%
%%%%%%%%%%%%%%%%%%%%%%%%%%%%%%%%%%

\documentclass[12pt,twoside,maitrise]{template/dms}
\usepackage{template/preamble}

%%%%%%%%%%%%%%%%%%%%%%%%%%%%%%%%%%%%%%%%%%%%%%%%%%%%%%%%%%%%
%%%%%%%%%%%%%%%%%%%%%%%%%%%%%%%%%%%%%%%%%%%%%%%%%%%%%%%%%%%%
%%%%%%%%%%                                     %%%%%%%%%%%%%
%%%%%%%%%% D é b u t    d u    d o c u m e n t %%%%%%%%%%%%%
%%%%%%%%%%                                     %%%%%%%%%%%%%
%%%%%%%%%%%%%%%%%%%%%%%%%%%%%%%%%%%%%%%%%%%%%%%%%%%%%%%%%%%%
%%%%%%%%%%%%%%%%%%%%%%%%%%%%%%%%%%%%%%%%%%%%%%%%%%%%%%%%%%%%
\begin{document}

%%
%% Voici des options pour annoter les différentes versions de votre
%% mémoire. La commande \brouillon imprime, au bas de chacune des pages, la
%% date ainsi que l'heure de la dernière compilation de votre fichier.
%%
%%\brouillon
%%
%%
%% \version est la version de votre manuscrit
%%
\version{1}

%%------------------------------------------------- %
%%              pages i et ii                       %
%%------------------------------------------------- %

%%%
%%% Voici les variables à définir pour les deux premières pages de votre
%%% mémoire.
%%%

\title{Évaluation de l'unicité écologique à grande étendue spatiale à l'aide de 
       modèles de répartition d'espèces}

\author{Gabriel Dansereau}

\copyrightyear{2021}

\department{Département de sciences biologiques}

\date{7 mai 2021} %Date du DÉPÔT INITIAL (ou du 2e dépôt s'il y a corrections majeures)

\sujet{sciences biologiques}
%%\orientation{orientation}%Ce champ est optionnel
%%
%% Voici les disciplines possibles (voir avec votre directeur):
%% \sujet{statistique},
%% \sujet{mathématiques}, \orientation{mathématiques appliquées},
%% \orientation{mathématiques fondamentales}
%% \orientation{mathématiques de l'ingénieur} et
%% \orientation{mathématiques appliquées}

\president{Anne-Lise Routier}

\directeur{Timothée Poisot}

\codirecteur{Pierre Legendre}         % s'il y a lieu
%%\codirecteurs{Nom du 2e codirecteur}         % s'il y a lieu

\membrejury{Élise Filotas}

%%\examinateur{Nom de l'examinateur externe}   %obligatoire pour la these

%% \membresjury{Deuxième membre du jury}  % s'il y a lieu

%%  \plusmembresjury{Troisième membre du jury}    % s'il y a lieu

 % Cette option existe encore, mais elle n'a plus sa place
 % dans la page titre. L'utiliser seulement si le directeur
 % insiste...
%%\repdoyen{Nom du représentant du doyen} %(thèse seulement)

%%
%% Fin des variables à définir. La commande \maketitle créera votre
%% page titre.

\maketitle

 % Pour générer la deuxième page titre, il faut appeler à nouveau \maketitle
 % Cette page est obligatoire.
\maketitle

%%------------------------------------------------- %
%%              pages iii                           %
%%------------------------------------------------- %

 % Les articles peuvent être en anglais, mais
 % les autres parties du document doivent être
 % en français. Il faut une permission pour
 % écrire l'ensemble de la thèse en anglais.
 % Consulter le guide de présentation des mémoires
 % et des thèses pour de l'information plus
 % précise et à jour.
\francais

\chapter*{Résumé}

La diversité bêta est une mesure essentielle pour décrire l'organisation de la biodiversité dans l'espace. Le calcul des contributions locales à la diversité bêta (LCBD), en particulier, permet d'identifier des sites à forte unicité écologique et montrant une diversité exceptionnelle au sein d'une région d'intérêt. Jusqu'à présent, l'utilisation des LCBD s'est restreinte à des échelles locales ou régionales avec un petit nombre de sites dont la composition en espèces est entièrement connue. Dans ce mémoire, j'ai examiné la variation des LCBD sur de grandes étendues spatiales, incluant des régions riches et pauvres en espèces. Pour ce faire, j'ai utilisé les arbres de régression additifs bayésiens (BARTs) pour prédire la répartition des parulines en Amérique du Nord à partir d'observations provenant de la base de données eBird. Mes résultats ont montré que la relation entre la richesse et les LCBD varie selon la région et l'étendue spatiale où celle-ci est mesurée et qu'elle est influencée par la proportion d'espèces rares dans les communautés. Les modèles de répartition d'espèces ont fourni des estimations fortement corrélées avec les données observées, mais montrant de l'autocorrélation spatiale. Les sites identifiés comme uniques sur de grandes étendues spatiales peuvent donc varier en fonction de la richesse régionale, de l'étendue totale et de la proportion d'espèces rares. Les modèles de répartition d'espèces peuvent être utilisés pour prédire l'unicité écologique sur de grandes étendues spatiales, ce qui pourrait permettre d'identifier des points chauds de diversité bêta et d'importantes cibles de conservation au sein de régions non échantillonnées.

\textbf{Mots clés}: diversité bêta, unicité écologique, contributions locales à la diversité bêta, modèles de répartition d'espèces, échelle spatiale étendue, eBird.

%%------------------------------------------------- %
%%              pages iv                            %
%%------------------------------------------------- %

\anglais
\chapter*{Abstract}

Beta diversity is an essential measure to describe the organization of biodiversity through space. The calculation of local contributions to beta diversity (LCBD), specifically, allows the identification of sites with high ecological uniqueness and exceptional diversity within a region of interest. To this day, LCBD indices have primarily been used on regional and smaller scales, with relatively few sites. Furthermore, their use is typically restricted to strictly sampled sites with known species composition, leading to gaps in spatial coverage on broad extents. Here, I investigated the variation of LCBD indices over broad spatial extents, including species-poor and species-rich regions, and examined their applicability for spatially continuous data and unsampled sites through species distribution modelling (SDM). To this aim, I used Bayesian additive regression trees (BARTs) to model warbler species composition on spatially continuous data based on observations recorded in the eBird database. My results showed that the relationship between richness and LCBD values varies according to the region and the spatial extent at which it is applied. It is also affected by the proportion of rare species in the community. Species distribution models provided highly correlated estimates with observed data, although spatially autocorrelated. Sites identified as unique over broad spatial extents may therefore vary according to regional richness, total extent size, and the proportion of rare species. Species distribution modelling can be used to predict ecological uniqueness over broad spatial extents, which could help identify beta diversity hotspots and important targets for conservation purposes in unsampled locations.

\textbf{Keywords}: beta diversity, ecological uniqueness, local contributions to beta diversity, species distribution modelling, broad spatial scale, eBird.

%%------------------------------------------------- %
%%        page v --- Table de matieres              %
%%------------------------------------------------- %

\francais
 % \cleardoublepage termine la page actuel et force TeX
 % a poussé les éléments flottant (fig., tables, etc.) sur
 % la page (normalement TeX les garde en suspend jusqu'à ce
 % qu'il trouve un endroit approprié). Avec l'option <twoside>,
 % la commande s'assure que la prochaine page de texte est sur
 % le recto, pour l'impression. On l'utilise ici
 % pour que TeX sache que la table des matières etc. soit
 % sur la page qui suit.
%% TABLE DES MATIÈRES
\cleardoublepage
\pdfbookmark[chapter]{\contentsname}{toc}  % Crée un bouton sur
                                           % la bar de navigation
\tableofcontents
 % LISTE DES TABLES
\cleardoublepage
\phantomsection  % Crée une section invisible (utile pour les hyperliens)
\listoftables
 % LISTE DES FIGURES
\cleardoublepage
\phantomsection
\listoffigures

%%%%%%%%%%%%%%%%%%%%%%%%%%%%%%%%%%%%%
%% LISTE DES SIGLES ET ABRÉVIATION %
%%%%%%%%%%%%%%%%%%%%%%%%%%%%%%%%%%%%%
%% Il est obligatoire, selon les directives de la FESP,
%% pour une thèse ou un mémoire d'avoir une liste des sigles et
%% des abréviations.  Si vous considérez que de telles listes ne seraient pas
%% pertinentes (si, par exemple, vous n'utilisez aucun sigle ou abré.), son
%% inclusion ou omission est laissé à votre discrétion.  En cas de doute,
%% parlez-en à votre directeur de recherche, le coadministrateur ou au/à la
%% bibliothécaire.
%%
%% Le gabarit inclut un exemple d'une liste « fait à la main ».  Il existe des outils
%% plus sophistiqués si vous devez inclure une multitude de sigles et abréviations.
%% Par exemple, le package <glossaries> peut faire des index élaborés.  Comme
%% son utilisation est technique, il n'y a pas d'exemple directement dans ce gabarit.
%% On invite les gens qui aurait à l'utiliser à lire la documentation officielle,
%% soit en allant sur https://www.ctan.org/, soit en tapant dans un terminal :
%%
%% texdoc glossaries
%%

\chapter*{Liste des sigles et des abréviations}
 % Option de colonnes: definir \colun ou \coldeux
%%% Exemple
%%% \def\colun{\bf} % Première colonne en gras
%%% Pour numéroté les entrées, on peut faire
%%% \newcount\abbrlist
%%% \abbrlist=0
%%% \def\plusun{\global\advance\abbrlist by 1\relax}
%%% \def\colun{\plusun\the\abbrlist. }
%%\def\coldeux{\relax}
\begin{twocolumnlist}{.2\textwidth}{.7\textwidth}
  BART  & Arbres de régression additifs bayésiens (\textit{Bayesian additive regression trees})\\
  BN & Réseaux bayésiens (\textit{Bayesian networks})\\
  BRT   & Arbres de régression fortifiés (\textit{Boosted regression trees})\\
  HMSC & Modélisation hiérarchique des communautés d’espèces (\textit{Hierarchical modelling of species communities})\\
  JSDM & Modèles conjoints de répartition d'espèces (\textit{Joint species distribution models})\\
  LCBD  & Contributions locales à la diversité bêta (\textit{Local contributions to beta diversity})\\
  MCMC & Méthode de Monte-Carlo par chaînes de Markov (\textit{Markov Chain Monte-Carlo method})\\
  MEM & Modèles macro-écologiques (\textit{Macroecological models})\\
  RF    & Forêts d'arbres décisionnels (\textit{Random Forests})\\
  SDM   & Modèles de répartition d'espèces (\textit{Species distribution models}) \\
  SESAM & Modélisation spatialement explicite des assemblages d’espèces (\textit{spatially explicit species assemblage modelling})\\
  S-SDM & Modèles de répartition d'espèces superposés (\textit{Stacked species distribution models})\\
  TSS & \textit{True skill statistic}\\
\end{twocolumnlist}
%% L'environnement <threecolumnlist> existe aussi pour trois colonnes.

%%------------------------------------------------- %
%%              pages vi                            %
%%------------------------------------------------- %

\chapter*{Remerciements}

Mon parcours à la maîtrise a été une expérience des plus stimulantes. La très grande partie du mérite pour ces expériences positives revient à ceux et celles que j’ai eu la chance de côtoyer.

Merci en premier à mes co-directeurs, Timothée et Pierre. Merci pour vos conseils, merci de m’avoir guidé dans mon parcours, merci d’avoir toujours eu des suggestions pour stimuler mes réflexions en recherche, et merci d’avoir su me dire quand il fallait m’arrêter. J’ai énormément appris grâce à vous et je vous en serai toujours reconnaissant.

Ensuite, merci à tous mes collègues avec qui j’ai eu la chance de vivre cette expérience. Sans vous, mon parcours aurait été beaucoup plus difficile. Je suis ressorti grandi du contact avec chacun et chacune d’entre vous. Merci à Daphnée, Eva et Mathilde pour vos conseils, votre expérience et votre présence très précieuse au début de ma maîtrise. Merci à Francis et Gracielle pour les nombreux projets stimulants, loufoques parfois, complexes à d’autres moments, mais toujours agréables en votre compagnie. Merci à tous les collègues du labo, Andréanne, Fares, Kiri, Marie-Andrée, Miléna, Norma, Philippe, Salomé, Sandrine, Tanya et Valentine.

Merci à Andrew d’avoir été tel un guide pour un voyageur intergalactique en recherche. Discuter avec toi donnait parfois l’impression d’aller jusqu’à la fin du monde, sur des sujets aussi variés que la vie, l’univers et tout le reste. Merci cependant pour tous les poissons, et sache que tes conseils, loin d’être généralement inoffensifs, ont toujours été utiles.

Merci au FRQNT et à BIOS², dont le soutien financier a permis la réalisation de ces travaux. 

Merci finalement à mes parents pour leur éternel soutien. Merci d’être présents dans tous les moments, difficiles comme heureux.

 %
 % Fin des pages liminaires.  À partir d'ici, les
 % premières pages des chapitres ne doivent pas
 % être numérotées
 %

\NoChapterPageNumber
\cleardoublepage


 % Il est recommandé que chaque article soit dans son propre .tex
 % Si la bibliographie de l'article doit apparaître à la fin de
 % l'article (plutôt qu'à la fin de la thèse), il obligatoire que
 % l'article soit dans son propre .tex
%%%%%%%%%%%%%%%%%%%%%%%%%%%%%%%%%%%%%%%%%%%%%%%%%%%%%%%%%%%%
%%%%%%%%%%%%%%%%                           %%%%%%%%%%%%%%%%%
%%%%%%%%%%%%%%%%  I N T R O D U C T I O N  %%%%%%%%%%%%%%%%%
%%%%%%%%%%%%%%%%                           %%%%%%%%%%%%%%%%%
%%%%%%%%%%%%%%%%%%%%%%%%%%%%%%%%%%%%%%%%%%%%%%%%%%%%%%%%%%%%

\francais
\chapter*{Introduction}
\input{assets/_introduction.tex}

%%%%%%%%%%%%%%%%%%%%%%%%%%%%%%%%%%%%%%%%%%%%%%%%%%%%%%%%%%%%
%%%%%%%%%%%%%%%%%%%%                   %%%%%%%%%%%%%%%%%%%%%
%%%%%%%%%%%%%%%%%%%%  A R T I C L E 1  %%%%%%%%%%%%%%%%%%%%%
%%%%%%%%%%%%%%%%%%%%                   %%%%%%%%%%%%%%%%%%%%%
%%%%%%%%%%%%%%%%%%%%%%%%%%%%%%%%%%%%%%%%%%%%%%%%%%%%%%%%%%%%
%% To change chapter header dynamically from french to english, use
%%\entetedynamique
\setcounter{corA}{0} % Pour recommencer à compter les def,
                     % theo, etc. à partir de 1
\setcounter{secnumdepth}{1}
\renewcommand{\thefigure}{\arabic{figure}}
\setcounter{figure}{0}
 % Pour écrire un article en français
% \francais
 % Pour écrire un article en anglais
\anglais
%% NOTE: La plupart des macros ont un nom en anglais.
%% P.ex. \adresse et \address fonctionnent et sont équivalents.
%% \revue=\journal
%% \auteur=\author
%% \titre=\title

 % Nom de la revue de publication
\revue{Global Ecology and Biogeography}
\article{Evaluating ecological uniqueness over broad spatial extents using species distribution modelling}

 % Contribution(s) personnelle(s) à l'article et rôle joué par tous les coauteurs
 %
 % Nécessaire seulement lorsque vous n'êtes pas seul·e auteur·e.
 % Les contributions peuvent apparaître ailleurs dans la thèse.
 % Si \contributions est laissé vide (p.ex. si vous effacez
 % celui ci-bas), aucune contributions ne seront générées sur
 % la page titre de l'article.
 %
 % REMARQUE : L'exemple est sous forme d'\itemize,
 % mais toutes les constructions \LaTeX\ sont permises.
 % La commande peut aussi contenir un simple petit texte.
 %
 % La commande admet une option [<entête>]
\contributions[]%[Mes contributions et le rôle des coauteurs]
{
    GD developed and performed the analyses and wrote the first version of the manuscript;

    TP developed a preliminary version of the analyses. 
    
    PL and TP provided guidance on the analyses and interpretation of the results and revised the manuscript.
    
    All authors read and approved the manuscript.\\[1cm]
}
%% Les contributions apparaîtrons après
%% \maketitle. Selon les goûts, il est
%% possible de mettre les contributions
%% avant, simplement en les écrivant
%% directement ici. Par exemple :
 % \section*{Contributions de <mon nom> et rôle joué par les coauteurs}
 % J'ai contribué en...
 %
 % Le rôle des coauteurs a été de...
 % \cleardoublepage

%%% INFORMATIONS POUR LA PAGE TITRE
 % Premier auteur·e et adresse
\auteur{Gabriel Dansereau}
% Deuxième auteur·e et adresse (si différente de la première)
\auteur{Pierre Legendre}
%%\adresse{4242 rue imaginaire\\ Universität von der gau\ss sche Zahlenebene}
%%
%% et ainsi de suite pour les autres auteurs
\auteur{Timothée Poisot}
\adresse{Département de sciences biologiques, Université de Montréal\\
  1375 avenue Thérèse-Lavoie-Roux, Montréal, QC, Canada H2V 0B3}
% \adresse{1205 Dr. Penfield Avenue, Montréal, QC, Canada, H3A 1B1 \\ Québec Centre for Biodiversity Science}

\maketitle

\begin{resume}{diversité bêta, unicité écologique, contributions locales à la diversité bêta, modèles de répartition d'espèces, échelle spatiale étendue, eBird}
  
  Objectif: La mesure des contributions locales à la diversité bêta (LCBD) permet d'identifier les sites à forte unicité écologique et ayant une composition en espèces exceptionnelle au sein d'une région d'intérêt. L'utilisation de cette mesure est cependant typiquement restreinte aux échelles locales et régionales avec un petit nombre de sites, puisqu'elle requiert des informations sur la composition complète des communautés parfois difficiles à acquérir à grande échelle spatiale. Dans cette étude, nous examinons comment la mesure des LCBD peut être utilisée pour prédire l'unicité écologique sur de grandes étendues spatiales à l'aide de modèles de répartition d'espèces et de données de science citoyenne.

  Lieu: Amérique du Nord
  
  Période de temps: Années 2000
  
  Taxon étudié: Parulidés
  
  Méthodes: Nous avons utilisé les arbres de régression additifs bayésiens (BARTs) pour prédire la répartition des parulines en Amérique du Nord à l'aide de données provenant de la base de données eBird. Nous avons ensuite calculé les valeurs de LCBD pour les données observées et prédites, puis nous avons examiné la différence par site à l'aide d'une comparaison directe, d'un test d'autocorrélation spatiale et de modèles de régression linéaire généralisés. Nous avons également examiné la variation de la relation entre les valeurs de LCBD et la richesse spécifique entre différentes régions et sur différentes étendues spatiales, ainsi que l'effet de la proportion d'espèces rares sur cette relation.
  
  Résultats: Nos résultats ont montré que la relation entre la richesse et les LCBD varie selon la région et l'étendue spatiale où celle-ci est mesurée et qu'elle est influencée par la proportion d'espèces rares dans la communauté. Les modèles de répartition d'espèces ont fourni des estimés fortement corrélés avec les données observées, mais qui montraient également de l'autocorrélation spatiale.
  
  Principales conclusions: Les sites identifiés comme uniques sur de grandes étendues spatiales peuvent varier en fonction de la richesse régionale, de l'étendue totale et de la proportion d'espèces rares dans la communauté. Les modèles de répartition d'espèces peuvent être utilisées pour prédire l'unicité écologique sur de grandes étendues spatiales, ce qui pourrait permettre d'identifier des points chauds de diversité bêta et d'importantes cibles de conservation au sein de régions non échantillonnées.

\end{resume}

\begin{abstract}{beta diversity, ecological uniqueness, local contributions to beta diversity, species distribution modelling, broad spatial scale, eBird}

  Aim: Local contributions to beta diversity (LCBD) can be used to identify sites with high ecological uniqueness and exceptional species composition within a region of interest. Yet, these indices are typically used on local or regional scales with relatively few sites, as they require information on complete community compositions difficult to acquire on larger scales. Here, we investigate how LCBD indices can be used to predict ecological uniqueness over broad spatial extents using species distribution modelling and citizen science data.
  
  Location: North America.
  
  Time period: 2000s.
  
  Major taxa studied: Parulidae. 
  
  Methods: We used Bayesian additive regression trees (BARTs) to predict warbler species distributions in North America based on observations recorded in the eBird database. We then calculated LCBD indices for observed and predicted data and examined the site-wise difference using direct comparison, a spatial autocorrelation test, and generalized linear regression. We also investigated the relationship between LCBD values and species richness in different regions and at various spatial extents and the effect of the proportion of rare species on the relationship. 
  
  Results: Our results showed that the relationship between richness and LCBD values varies according to the region and the spatial extent at which it is applied. It is also affected by the proportion of rare species in the community. Species distribution models provided highly correlated estimates with observed data, although spatially autocorrelated.
  
  Main conclusions: Sites identified as unique over broad spatial extents may vary according to the regional richness, total extent size, and the proportion of rare species. Species distribution modelling can be used to predict ecological uniqueness over broad spatial extents, which could help identify beta diversity hotspots and important targets for conservation purposes in unsampled locations.

\end{abstract}

% \section{Introduction}
%%
%% Le reste de l'article...
%%

\input{assets/_article1.tex}

% Exemple de citation~: Consultez le \LaTeX\ companien
% de Mittelbach {\it et al\/}\@. \citeyearpar{exemple}.

 % Pour générer la bibliographie à la fin
 % de l'article, utiliser la commande de la
 % classe <dms> \sectionbibliography{<nom du .bib>}.
 % Il y a aussi la possibilité d'utiliser le package
 % <chapterbib>, auquel cas on utilise simplement
 % \bibliography normalement.
 %
 % IMPORTANT : Dans tous les cas, il faut faire
 %    pdflatex these
 %    bibtex chapitre1
 %    bibtex chapitre2
 %    .
 %    .
 %    .
 %    bibtex chapitreN
 %    pdflatex these
 %    pdflatex these
 %
 % où <these> est le nom du .tex principal
 % (qui contient le \documentclass).
 % bibtex a besoin du .aux de chapitre1 et
 % non du .tex. Il est parfois nécessaire
 % d'effacer le .aux et de recommencer la
 % compilation du début.
%%\bibliographystyle{plain} % style plain anglais ou
% \bibliographystyle{plain-fr} % style plain francais
% \bibliographystyle{plainnat-fr}
% \bibliographystyle{apalike-fr}
% \bibliographystyle{apacite}
% \printbibliography
%%\bibliographystyle{<style>} % autre
% \sectionbibliography{references.bib} %Donner le nom du .bib

\endinput
%%
%% End of file `article1.tex'.


%%%%%%%%%%%%%%%%%%%%%%%%%%%%%%%%%%%%%%%%%%%%%%%%%%%%%%%%%%%%
%%%%%%%%%%%%%%%%%%%%                   %%%%%%%%%%%%%%%%%%%%%
%%%%%%%%%%%%%%%%%%%%  CONCLUSION  %%%%%%%%%%%%%%%%%%%%%
%%%%%%%%%%%%%%%%%%%%                   %%%%%%%%%%%%%%%%%%%%%
%%%%%%%%%%%%%%%%%%%%%%%%%%%%%%%%%%%%%%%%%%%%%%%%%%%%%%%%%%%%

\francais
\chapter*{Conclusion}
\input{assets/_conclusion.tex}

%%
%% etc.

 % S'il y a une bibliographie pour tout le document, on peut
 % utiliser les commandes suivantes. À noter que le style est
 % laisser au choix de l'auteur·e. (Il est même possible
 % d'utiliser <natbib>).
 % Il est possible d'avoir une bibliographie pour chaque
 % chapitre. Consulter l'article en exemple pour voir
 % comment faire.
%%\bibliographystyle{plain-fr}
%%\bibliography{<fichier.bib>}

\francais

\printbibliography

 % Pour les annexes :
\appendix
 % Les annexes se font comme les chapitres. Le fichier
 % commence par \francais ou \anglais et ensuite
 % \chapter{..}. Le reste est pareil à un chapitre normal.
%%%%%%%%%%%%%%%%%%%%%%%%%%%%%%%%%%%%%%%%%%%%%%%%%%%%%%%%%%%%
%%%%%%%%%%%%%%%%%%%%                   %%%%%%%%%%%%%%%%%%%%%
%%%%%%%%%%%%%%%%%%%%   A N N E X E 1   %%%%%%%%%%%%%%%%%%%%%
%%%%%%%%%%%%%%%%%%%%                   %%%%%%%%%%%%%%%%%%%%%
%%%%%%%%%%%%%%%%%%%%%%%%%%%%%%%%%%%%%%%%%%%%%%%%%%%%%%%%%%%%

\setcounter{secnumdepth}{0}
\renewcommand{\thefigure}{A\arabic{figure}}
\setcounter{figure}{0}

 \anglais 

 \revue[This article was published in ]{Journal of Open Source Software}
 \article{SimpleSDMLayers.jl and GBIF.jl: A Framework for Species Distribution Modelling in Julia}

\contributions[]
{
    GD and TP developed the software. GD wrote the manuscript.
}

\auteur{Gabriel Dansereau}
\auteur{Timothée Poisot}
\adresse{Département de sciences biologiques, Université de Montréal\\
  1375 avenue Thérèse-Lavoie-Roux, Montréal, QC, Canada H2V 0B3}

\maketitle

\input{assets/_appendix_joss.tex}

%%
%% etc.

\end{document}

\endinput
%%
%% End of file `gabaritTPA.tex'.
