%%
%% This is file `gabaritTPA.tex',
%% generated with the docstrip utility.
%%
%% The original source files were:
%%
%% dms.dtx  (with options: `TPA,gabarit')
%% Example TeX file for the documentation
%% of the jurabib package
%% Copyright (C) 1999, 2000, 2001 Jens Berger
%% See dms.ins  for the copyright details.
%% 
%%% ====================================================================
%%%  @LaTeX-file{
%%%     filename        = "dms.dtx",
%%%     author    = "Nicolas Beauchemin, Damien Rioux-Lavoie, Victor Fardel, Jonathan Godin",
%%%     copyright = "Copyright (C) 2000 , DMS
%%%                  all rights reserved.  Copying of this file is
%%%                  authorized only if either:
%%%                  (1) you make absolutely no changes to your copy,
%%%                  including name; OR
%%%                  (2) if you do make changes, you first rename it
%%%                  to some other name.",
%%%     address   = "Département de Mathématiques et de Statistique",
%%%     telephone = "514-343-6705",
%%%     FAX       = "514-343-5700",
%%%     email     = "aide@dms.umontreal.ca (Internet)",
%%%     keywords  = "latex, amslatex, ams-latex, theorem",
%%%     abstract  = " Ce fichier est un package conçu pour être
%%%                  utilisé avec la version de LaTeX2e 1995/06/01. Il
%%%                  est prévue pour la classe ``amsbook''. Il en
%%%                  modifie le format des pages, l'entête des
%%%                  sections, etc, afin d'être  conforme au modèle de
%%%                  mémoire de maîtrise de l'Université de
%%%                  Montréal. Finalement ce fichier est grandement
%%%                  inspiré du fichier amsclass.dtx.",
%%%     docstring = "The checksum field contains: CRC-16 checksum,
%%%                  word count, line count, and character count, as
%%%                  produced by Robert Solovay's checksum utility."}
%%%  ====================================================================

%% Pour voir les accents de ce fichier, assurez-vous que votre
%% éditeur de texte lise le fichier en utf-8!

%% La classe <dms> est construite au-dessus de <amsbook>, donc
%% <amsmath>, <amsfonts> et <amsthm> sont automatiquement chargés.
\documentclass[12pt,twoside,phd]{dms}
\usepackage[utf8]{inputenc} %Obligatoires
\usepackage[T1]{fontenc}    %

%% <lmodern> incorpore les fontes en T1, pour
%% faciliter le dépôt final. Ceci n'est pas la
%% seule option :
%%  1. Si cm-super est installé, vous pouvez enlever <lmodern>
%%     (à ce moment, la police est un peu plus fidèle
%%      au Computer Modern orginal);
%%  2. Si vous avez une police préférée, par exemple,
%%     <times> ou <euler> ou <mathpazo> (et bien d'autres),
%%     alors vous pouvez remplacer <lmodern> ci-bas.
%% Par contre, si vous faîtes face à un problème d'encapsulation
%% lors dépôt final, il se peut que la solution soit d'utiliser <lmodern>.
%% (Parfois le problème est au niveau de l'installation, donc
%%  essayez de compiler sur un autre ordinateur sur lequel vous êtes
%%  certain·e que l'installation est bonne.)
\usepackage{lmodern}
\DeclareSymbolFont{largesymbols}{OMX}{cmex}{m}{n}

%% TEXTBOX EXPERIEMENTS

\usepackage[most]{tcolorbox}


\newtcbtheorem{Summary}{\bfseries Box}{enhanced,drop shadow={black!50!white},
  coltitle=black,
  top=0.3in,
  attach boxed title to top left=
  {xshift=1.5em,yshift=-\tcboxedtitleheight/2},
  boxed title style={size=small,colback=pink}
}{summary}

\newtcolorbox[auto counter]{summary}[1][]{title={\bfseries Box~\thetcbcounter},enhanced,drop shadow={black!50!white},
  coltitle=black,
  top=0.3in,
  attach boxed title to top left=
  {xshift=1.5em,yshift=-\tcboxedtitleheight/2},
  boxed title style={size=small,colback=pink},#1}

%% FUNNY LETTERS

\usepackage{mathrsfs}

%% Il n'est pas nécessaire d'utiliser <babel>, car
%% les commandes intégrées par la classe <dms>
%% \francais et \anglais font le travail. Néanmoins,
%% certains autres packages nécessitent <babel> (comme
%% <natbib>), donc simplement enlever les % devant <babel>
%% dans ce cas. Attention! Certains packages sont sensibles
%% à l'ordre dans lequel ils sont chargés.
%%\francais % or
\anglais
%%
%%\usepackage[english,frenchb]{babel}

 % ENGLISH OPTION
 % If you call \anglais here before the \begin{document},
 % all the chater's header will be in english, even if you
 % call \francais. To change this, use
 % \entetedynamique

%% La commande \sloppy peut avoir des effets étranges sur les
%% lignes de certains paragraphes.  Dans ce cas, essayez \fussy
%% qui suppresse les effets de \sloppy.
%% (\fussy est normalement le comportement par défaut.)
%% On redéfinit \sloppy, pour tenter de réduire les comportements
%% étranges. Le seul changement apporté à la version originale
%% est la valeur de \tolerance.
\def\sloppy{%
  \tolerance 500%  %9999 dans LaTeX ordinaire, mauvaise idée.
  \emergencystretch 3em%
  \hfuzz .5pt
  \vfuzz\hfuzz}
\sloppy   %appel de \sloppy pour le document
%%\fussy  %ou \fussy

%% Packages utiles.
\usepackage{graphicx,amssymb,subfigure,icomma}
%% icomma       permet d'écrire les nombres décimaux en
%%                  français (p.ex. 1,23 plutôt que 1.23)
%% subfigure    simplifie l'inclusion de figures côtes-à-côtes

%% Packages parfois utiles.
%%\usepackage{dsfont,mathrsfs,color,url,verbatim,booktabs}
%% dsfont       symboles mathématiques \mathds
%% mathrsfs     plus de symboles mathématiques \mathscr
%% color        pour utiliser des couleurs (comparer avec <xcolor>)
%% url          permet l'écriture d'url
%% verbatim     pour écrire du code ou du texte tel quel
%% booktabs     plus de macros pour faire les tableaux
%%                  (voir documentation du package)

%% pour que la largeur de la légende des figures soit = \textwidth
\usepackage[labelfont=bf, width=\linewidth]{caption}

%% les 3 lignes suivante servent à l'affichage de l'index
%% dans le visionneur de pdf. <hyperref> et <bookmark>
%% devraient être les dernier package a être chargé,
%% donc chargez vos packages avant.
\usepackage{hyperref}  % Ajoute les hyperlien
\hypersetup{colorlinks=true,allcolors=black}
\usepackage{hypcap}   % Corrige la position du lien pour les images
\usepackage{bookmark} % Remédie à des petits problème
                      % de <hyperref> (important qu'il
                      % apparaisse APRÈS <hyperref>)

  % Enlever les commentaires du prochaine \hypersetup et
  % le remplir avec l'information pertinente.
  % Ceci ajoute des « méta-données » au pdf.  C'est optionnel,
  % mais recommandé. Vous pouvez voir ces méta-données en
  % ouvrant un visionneur de pdf et en cherchant les propriétés
  % du pdf. (Vous pouvez aussi tapez ' pdfinfo <nom-du-pdf> '
  % dans un terminal.) Ces données sont utiles, par exemple,
  % pour augmenter les chances qu'un algorithme de recherche
  % trouve votre document sur Internet, une fois diffusé.
%%\hypersetup{
%%  pdftitle = {Titre de la thèse / du mémoire},
%%  pdfauthor = {auteur·e},
%%  pdfsubject = {Ex: Transformation de Fourier ; régressions linéaires ; ... },
%%  pdfkeywords = {Ex: mathématiques, statistiques, groupes, variables aléatoires,...}
%%}

%% Définition des environnements utiles pour un mémoire scientifique.
%% La numérotation est laissée à la discrétion de l'auteur·e. L'exemple
%% illustré ici produit « Définition x.y.z » à l'extérieur d'un article
%%   x = no. chapitre
%%   y = no. section
%%   z = no. définition
%% et « Définition x » à l'intérieur d'un article
%%   x = no. définition
%% Les numérotations des corollaires, définitions, etc.
%% se font de façon successive.
%%
%% Les macros \<type>name sont telles qu'ils suivent
%% la langue actuelle. (P.ex. si \francais est utilisé,
%% alors \begin{theo} va faire un Théorème et si \anglais
%% est utilisé, \begin{theorem} fera un Theorem.)
%%
  % Environnement à utiliser à l'extérieur des articles
\newtheorem{cor}{\corollaryname}[section]
\newtheorem{deff}[cor]{\definitionname}
\newtheorem{ex}[cor]{\examplename}
\newtheorem{lem}[cor]{\lemmaname}
\newtheorem{prop}[cor]{Proposition}
\newtheorem{rem}[cor]{\remarkname}
\newtheorem{theo}[cor]{\theoremname}

  % Environnement à utiliser à l'intérieur des articles
\newtheorem{corA}{\corollaryname}
\newtheorem{deffA}[corA]{\definitionname}
\newtheorem{exA}  [corA]{\examplename}
\newtheorem{lemA} [corA]{\lemmaname}
\newtheorem{propA}[corA]{Proposition}
\newtheorem{remA} [corA]{\remarkname}
\newtheorem{theoA}[corA]{\theoremname}
%% IMPORTANT : Il faut faire \setcounter{corA}{0}
%% au début d'un article pour recommancer à compter à 1.
%%
%% NOTE : Il peut être commode de redéfinir \the<type> pour
%% obtenir la numérotation désirée. Par exemple, pour
%% que les corollaires soit numérotés #article.#section.#sous-section,
%% on fait
%% \renewcommand\thecorA{\thepart.\thesubsection.\arabic{corA}}

%%%
%%% Si vous préférez que les corollaires, définitions, théorèmes,
%%% etc. soient numérotés séparément, utilisez plutôt un bloc de
%%% commandes de la forme :
%%%

%%\newtheorem{cor}{\corollaryname}[section]
%%\newtheorem{deff}{\definitionname}[section]
%%\newtheorem{ex}{\examplename}[section]
%%\newtheorem{lem}{\lemmaname}[section]
%%\newtheorem{prop}{Proposition}[section]
%%\newtheorem{rem}{\remarkname}[section]
%%\newtheorem{theo}{\theoremname}[section]

%%
%% Numérotation des équations par section
%% et des  tableaux et figures par chapitre.
%% Ceci peut être modifié selon les préférences de l'utilisateur.
\numberwithin{equation}{section}
\numberwithin{table}{chapter}
\numberwithin{figure}{chapter}

%%
%% Si on veut faire un index, il faut décommenter la ligne
%% suivante. Ajouter des mots à l'index avec la commande \index{mot cle} au
%% fur et à mesure dans le texte.  Compiler, puis taper la commande
%% makeindex pour creer les indexs.  Après une nouvelle compilation,
%% vous aurez votre index.
%%

%%\makeindex

%% Il est obligatoire d'écrire à double interligne
%% ou à interligne et demi. On peut soit utiliser
%% le package <setspace> ou \baselinestretch.
%% Le package a tendance a créé des grands espaces blancs,
%% le gabarit décourage son utilisation, mais il en
%% reste à la discrétion de l'utilisateur·e.
%% \usepackage[onehalfspacing]{setspace}
 % ou
\renewcommand{\baselinestretch}{1.286} %Interligne et demi (environ 18pt (12pt+6pt) entre les lignes)

%%%%%%%%%%%%%%%%%%%%%%%%%%%%%%%%%%%%%%%%%%%%%%%%%%%%%%%%%%%%
%%%%%%%%%%%%%%%%%%%%%%%%%%%%%%%%%%%%%%%%%%%%%%%%%%%%%%%%%%%%
%%%%%%%%%%                                     %%%%%%%%%%%%%
%%%%%%%%%% D é b u t    d u    d o c u m e n t %%%%%%%%%%%%%
%%%%%%%%%%                                     %%%%%%%%%%%%%
%%%%%%%%%%%%%%%%%%%%%%%%%%%%%%%%%%%%%%%%%%%%%%%%%%%%%%%%%%%%
%%%%%%%%%%%%%%%%%%%%%%%%%%%%%%%%%%%%%%%%%%%%%%%%%%%%%%%%%%%%
\begin{document}

%%
%% Voici des options pour annoter les différentes versions de votre
%% mémoire. La commande \brouillon imprime, au bas de chacune des pages, la
%% date ainsi que l'heure de la dernière compilation de votre fichier.
%%
%%\brouillon
%%
%%
%% \version est la version de votre manuscrit
%%
\version{1}
\pagenumbering{roman}

%%------------------------------------------------- %
%%              pages i et ii                       %
%%------------------------------------------------- %

%%%
%%% Voici les variables à définir pour les deux premières pages de votre
%%% mémoire.
%%%

\title{ART OFFICIAL AGE: Understanding the common properties of species interaction networks across space using fancy maths}

\author{Tanya Strydom}

\copyrightyear{2023}

\department{Sciences biologiques}

\date{\today} %Date du DÉPÔT INITIAL (ou du 2e dépôt s'il y a corrections majeures)

\sujet{Sciences biologiques}
%%\orientation{orientation}%Ce champ est optionnel
%%
%% Voici les disciplines possibles (voir avec votre directeur):
%% \sujet{statistique},
%% \sujet{mathématiques}, \orientation{mathématiques appliquées},
%% \orientation{mathématiques fondamentales}
%% \orientation{mathématiques de l'ingénieur} et
%% \orientation{mathématiques appliquées}

\president{Nom du président du jury}

\directeur{Timothée Poisot}

%%\codirecteur{Nom du 1er codirecteur}         % s'il y a lieu
%%\codirecteurs{Nom du 2e codirecteur}         % s'il y a lieu

\membrejury{Nom du membre de jury}

%%\examinateur{Nom de l'examinateur externe}   %obligatoire pour la these

%% \membresjury{Deuxième membre du jury}  % s'il y a lieu

%%  \plusmembresjury{Troisième membre du jury}    % s'il y a lieu

 % Cette option existe encore, mais elle n'a plus sa place
 % dans la page titre. L'utiliser seulement si le directeur
 % insiste...
%%\repdoyen{Nom du représentant du doyen} %(thèse seulement)

%%
%% Fin des variables à définir. La commande \maketitle créera votre
%% page titre.

%% Pour mettre bouton qui mène à la page titre
%% dans le visionneur de pdf. Peut être enlever.
\pdfbookmark[chapter]{Couverture}{PageUn}

\maketitle

 % Pour générer la deuxième page titre, il faut appeler à nouveau \maketitle
 % Cette page est obligatoire.
\maketitle

%%------------------------------------------------- %
%%              pages iii                           %
%%------------------------------------------------- %

 % Les articles peuvent être en anglais, mais
 % les autres parties du document doivent être
 % en français. Il faut une permission pour
 % écrire l'ensemble de la thèse en anglais.
 % Consulter le guide de présentation des mémoires
 % et des thèses pour de l'information plus
 % précise et à jour.
\francais

\chapter*{Résumé}

...sommaire et mots clés en français...

%%------------------------------------------------- %
%%              pages iv                            %
%%------------------------------------------------- %

\anglais
\chapter*{Abstract}

General intro about common properties of species interaction 
networks across space but also the challenges because we don't
really have much info to work with\\

The first chapter delves into thinking about the complexity
of networks and the different ways we might choose to define
complexity. More specifically here my collaborators and I
challenge the more traditional 'behavioural' measures of 
complexity (\emph{e.g.} nestedness, spectral radius, and
connectance) by presenting singular value decomposition
(SVD) entropy as an alternative ('physical') measure of
complexity. Taking a physical approach to defining complexity
allows us to think more about the information contained
within a network as opposed to the emerging properties
thereof. Interestingly SVD entropy reveals that bipartite
networks are highly complex and do not necessarily conform
to the idea that complexity begets stability. Although it
does not negate the value in taking a structural approach to
defining complexity SVD entropy as a measure of complexity
does tell us that networks contain a lot of information.
This is promising in the context of network prediction since
this information can make the process of prediction simpler
and more attainable.\\

Chapters three through five really delve into the idea of
network prediction, both in terms of the feasibility thereof
as well as  'prediction in action' by setting out to create
a metaweb for Canadian mammals. Chapter two is the product
of a working group consisting mainly of grad students and
sets out to map out what tools we can use to help us with
making network predictions. Here we look at the different
methodological considerations, and data sources needed for
making feasible predictions as well as forecasting. Chapter
three presents a 'tangible' approach to network prediction,
specifically the use of transfer learning to build a
probabilistic metaweb for Canadian mammals. The transfer
learning framework builds on the idea that we can take the
knowledge that we gain from solving a known problem and 
transfer this through a shared medium to solve a closely
related but unknown problem. In the context of this work we
used the (known) European metaweb to predict a metaweb
for Canadian species based on their phylogenetic relatedness
to European species as the transfer medium. What makes this
work particularly exciting is that despite the low number
of species shared between these two regions we are able to
recover a large percentage of interactions, all at a very
low computational cost. This sets up this methodology as
an ideal candidate for potentially beginning to fill in
the gaps on the global map and setting us up to start
asking large scale questions about networks. The final
chapter concerned with network prediction acts as a
companion piece to the work from the previous chapter. This
provides a more in-depth description in to methodology
used as well as a broader, more scoping, outlook as to
the potential uses and modification of embedding techniques\\

The final chapter present the Julia package SpatialBoundaries.jl.
This package allows the user to implement the spatial
wombling algorithm for both data arranged uniformly or
randomly across space. Because the spatial wombling
algorithm focuses on both the gradient as well as the
direction of change for the given landscape it can be
used both for detecting boundaries in the traditional
sense \emph{i.e.} regions of abrupt change as well as
a more nuanced look at at the direction of changes (\emph{e.g.}
do they all follow a specific direction?). This approach
could be a beneficial way with which to think about
questions which relate to boundary detection for networks
across space and how these do (or do not) relate to
environmental boundaries. Some conceptual ideas relating to
this are presented in Appendix \emph{TODO}

%%------------------------------------------------- %
%%        page v --- Table de matieres              %
%%------------------------------------------------- %

\anglais
 % \cleardoublepage termine la page actuel et force TeX
 % a poussé les éléments flottant (fig., tables, etc.) sur
 % la page (normalement TeX les garde en suspend jusqu'à ce
 % qu'il trouve un endroit approprié). Avec l'option <twoside>,
 % la commande s'assure que la prochaine page de texte est sur
 % le recto, pour l'impression. On l'utilise ici
 % pour que TeX sache que la table des matières etc. soit
 % sur la page qui suit.
%% TABLE DES MATIÈRES
\cleardoublepage
\pdfbookmark[chapter]{\contentsname}{toc}  % Crée un bouton sur
                                           % la bar de navigation
\tableofcontents
 % LISTE DES TABLES
\cleardoublepage
\phantomsection  % Crée une section invisible (utile pour les hyperliens)
\listoftables
 % LISTE DES FIGURES
\cleardoublepage
\phantomsection
\listoffigures

%%%%%%%%%%%%%%%%%%%%%%%%%%%%%%%%%%%%%
%% LISTE DES SIGLES ET ABRÉVIATION %
%%%%%%%%%%%%%%%%%%%%%%%%%%%%%%%%%%%%%
%% Il est obligatoire, selon les directives de la FESP,
%% pour une thèse ou un mémoire d'avoir une liste des sigles et
%% des abréviations.  Si vous considérez que de telles listes ne seraient pas
%% pertinentes (si, par exemple, vous n'utilisez aucun sigle ou abré.), son
%% inclusion ou omission est laissé à votre discrétion.  En cas de doute,
%% parlez-en à votre directeur de recherche, le coadministrateur ou au/à la
%% bibliothécaire.
%%
%% Le gabarit inclut un exemple d'une liste « fait à la main ».  Il existe des outils
%% plus sophistiqués si vous devez inclure une multitude de sigles et abréviations.
%% Par exemple, le package <glossaries> peut faire des index élaborés.  Comme
%% son utilisation est technique, il n'y a pas d'exemple directement dans ce gabarit.
%% On invite les gens qui aurait à l'utiliser à lire la documentation officielle,
%% soit en allant sur https://www.ctan.org/, soit en tapant dans un terminal :
%%
%% texdoc glossaries
%%

\chapter*{List of abbreviations}
 % Option de colonnes: definir \colun ou \coldeux
%%% Exemple
%%% \def\colun{\bf} % Première colonne en gras
%%% Pour numéroté les entrées, on peut faire
%%% \newcount\abbrlist
%%% \abbrlist=0
%%% \def\plusun{\global\advance\abbrlist by 1\relax}
%%% \def\colun{\plusun\the\abbrlist. }
%%\def\coldeux{\relax}
\begin{twocolumnlist}{.2\textwidth}{.7\textwidth}
  RDPG & Random Dot Product Graph\\
  SVD & Singular Value Decomposition\\
\end{twocolumnlist}
%% L'environnement <threecolumnlist> existe aussi pour trois colonnes.

%%------------------------------------------------- %
%%              pages vi                            %
%%------------------------------------------------- %

\newpage
\begin{center}
\emph{For those with the messy notebooks.\\
To those always seeking a way}
\end{center}

\chapter*{Acknowledgements}

The various projects that make up this thesis were conducted on
land within the traditional unceded territory of the Saint 
Lawrence Iroquoian, Anishinabewaki, Mohawk, Huron-Wendat, and
Omàmiwininiwak nations.\\

To mom and dad. Thanks for letting me draw on the windows when
the notebook space just wasn't enough. I'm sure the funny letters
will wash off one day...\\

To my many, many (awesome and all around great) collaborators. 
To the roadmap team; Michael Catchen, Francis Banville,
Dominique Caron, Gabriel Dansereau, Philippe Desjardins-Proulx
Norma Forero-Muñoz, Gracielle Higino, Benjamin Mercier,
Andrew Gonzalez, Dominique Gravel, and Laura Pollock --- thank
you for bringing your diverse scientific backgrounds, thoughts,
and ideas to the table. To the metaweb team; Salomé Bouskila,
Francis Banville, Ceres Barros, Dominique Caron, Maxwell Farrell,
Marie-Josée Fortin, Victoria Hemming, Benjamin Mercier,
Laura Pollock, and Rogini Runghen --- thanks for the (seemingly)
endless rounds of feedback and tweaking to make the manuscripts
more reader friendly and robust - I \emph{think} we're almost 
there! A special shout out to Giulio Dalla Riva for bringing
the endless energy and enthusiasm when it comes to anything SVD
related!\\

To my advisor Timothée Poisot. Thank you for taking in the field
ecologist who wanted to dip their toes into the world of thinking
boxes and species interaction networks. (Turns out that the need to
have a little cry before continuing with work is not unique to 
field work but also extends to trying to make the code go brrr).
Thanks for affording me the space to grow not only as a scientist
but also as an artist (\emph{sensu lato}). The chance to experiment
with visual ways to communicate our science may not necessarily
have resulted in more succinct manuscripts but it has for sure
changed the way I interact with my research and makes me think
about the ways we communicate our science maybe a bit too much!\\

To the (past and present) members of the Poisot Lab. The last
few years may not have been the best environment for nurturing
a collaborative environment but you made the best of it, but, 
despite the lag-y online calls and weird time differences you
managed to make 'lab-life' feel somewhat normal. Thanks (or 
should I say \emph{merci}) for being not only a source of 
scientific debate and discussion but also a first port of call
when trying to navigate university administration. Sorry for 
always running overtime in my 1:1's (despite my best efforts).
Look after Ahsoka!\\

A special nod to Gracielle Higino. Thanks for always bringing
the \emph{ENERGY} and for your continued mentorship and 
commitment to making science a KINDER place.\\

Thank you to those that are the driving force behind the Living
Data Project and BIOS$^2$ training programs. The exposure and
training opportunities related to 'real world' science outside
of school has been invaluable.\\

This work would not have been possible without funding from the
Courtois Foundation, the Canadian Institute for Ecology \& Evolution
(CIEE), the Viral Emergence Research Initiative (VERENA), and support
provided by Calcul Québec (www.calculquebec.ca) and Compute Canada
(www.computecanada.ca).

 %
 % Fin des pages liminaires.  À partir d'ici, les
 % premières pages des chapitres ne doivent pas
 % être numérotées
 %

\NoChapterPageNumber
\cleardoublepage
\pagenumbering{arabic}


 % Il est recommandé que chaque article soit dans son propre .tex
 % Si la bibliographie de l'article doit appaître à la fin de
 % l'article (plutôt qu'à la fin de la thèse), il obligatoire que
 % l'article soit dans son propre .tex
%%%%%%%%%%%%%%%%%%%%%%%%%%%%%%%%%%%%%%%%%%%%%%%%%%%%%%%%%%%%
%%%%%%%%%%%%%%%%                           %%%%%%%%%%%%%%%%%
%%%%%%%%%%%%%%%%  I N T R O D U C T I O N  %%%%%%%%%%%%%%%%%
%%%%%%%%%%%%%%%%                           %%%%%%%%%%%%%%%%%
%%%%%%%%%%%%%%%%%%%%%%%%%%%%%%%%%%%%%%%%%%%%%%%%%%%%%%%%%%%%
 % Utilisez la macro de langue appropriée.
 % Noter que toutes les parties du document,
 % à part les articles, doivent être en français.
 % Pour rédiger une thèse en anglais, il faut
 % une permission. Consulter le guide de présentation
 % des mémoires et des thèses pour de l'information
 % plus détaillé et à jour.
%%\francais   %ou
\anglais
\chapter*{Introduction}

\section{A case for tools and methods}

The way that species interact with one another provides a `point of
departure' from which to study or understand biodiversity and the
environment at a range of scales \cite{Jordano2016ChaEco}. Ranging from
understanding how interactions can shape and drive population dynamics,
the maintenance and functioning of ecosystems, as well as long-term
evolutionary dynamics \cite{Landi2018ComSta, Albrecht2018PlaAni}.
Species interactions (and the resulting networks) can be formalised
and viewed under the lens of graph theory \cite{Dale2010GraSpa} - with
species being nodes and interactions being edges. This provides us with
a robust framework built on a mathematical foundation from which to approach
network analysis and quantify various measures of network structure and
behaviour \cite{Delmas2019AnaEco}.

In the process of assembling ecological networks as graphs we are also
`encoding' an `ecological fingerprint' for that community. This raises
the question of how far we can take the the idea of `decoding' networks
by leveraging the mathematical framework to better understand the
information that they contain. In particular by leaning on the
mathematical properties (and the ecological information they represent)
to make network predictions, and as a means to provide us with more
information as to how networks may vary over time or spatial scales.

Although the field of network ecology might have a strong conceptual and
theoretical basis from which to work with we are still at somewhat of a
loss when is comes to our ability to leverage this framework to make any
generalised or macroecological conclusions about the properties of networks
over larger geographic scales (although see \cite{Baiser2019Ecogeographical, Pinheiro2023Latitudinal} who explicitly try and tie networks to classical macroecological theories/laws).
This limited understanding can (at least in a large part) be attributed to
the sparse global coverage of data \cite{Poisot2021GloKno, Cameron2019UneGlo},
which itself is driven by the immense
challenges associated with observing and recording interactions in the field 
\cite{Bennett2019PotPit, Jordano2016SamNet}. Given the limited feasibility
of being able to curate interaction datasets in a way that will result in
a global coverage it makes sense to turn to predictive methods as a way to
begin filling in the 'gaps' of the global map of interaction data. Although
this may seem a daunting task we can lean on the mathematical formalisation
and the information that networks contain to make this a possibility, once
we have crossed that bridge we may then find ourselves in a position to be
able to ask more global-minded questions.

This pipeline from prediction to global questions is shown in \autoref{fig:plan}
and is the mainstay of this thesis document \emph{i.e.,} the thesis itself
can be thought of as two parts. The first part is addressing
the need for predictive tools and discussing as well as developing methods we
can use to begin filling in the global map. The second phase of the thesis briefly touches on some new 'tools' we can use when we start to think about large scale questions pertaining to network properties, specifically
the question of network complexity and how the definition thereof matters, as well
as network boundaries.

\begin{figure}[h]
    \centering
    \includegraphics[width=\textwidth]{figures/nullmodel_richness.png}
    \caption{One of the biggest factors limiting our ability to ask global questions about ecological networks is the lack of global data. This figure provides a high-level overview of how the development and adoption of predictive methods will equip us to begin asking and answering large-scale questions. Two of these 'global questions' that are asked in this thesis are shown}
    \label{fig:plan}
\end{figure}

\subsection{Prediction for gap-filling}

Current methods are often conceptualised around and focused on a single facet of the
larger process at play such phylogenetic matching
\cite{Pomeranz2018InfPre, Elmasri2020HieBay}, or functional traits
\cite{Bartomeus2016ComFra}. Although recent applications of ensemble
modelling \cite{Becker2020PreWil}, and discussions on the potential of
machine learning methods \cite{Desjardins-Proulx2019ArtInt} show promise
in addressing methodological constraints to prediction and the growth of
open tools and data may mitigate some data constraints in the coming
years. However, we still lack a clear path forward or research agenda as
to how we can maximise and integrate these resources.

The task of trying to predict networks is discussed in chapters \ref{Roadmap}
and \ref{Perspectives}, where my collaborators and myself map out and
discuss the the methodological considerations when trying to approach the task
of network prediction. Chapter~\ref{Roadmap} provides a more scoping discussion
on these methods, whereas Chapter~\ref{Perspectives} represents a more detailed
discussion on the prospect of using graph embedding and transfer learning
for network prediction. These specific methods are also the framework presented
and used in Chapter~\ref{Foodweb}. This section acts as a 'proof-of-concept'
showcasing that the task of network prediction is both attainable and
capable of producing ecologically plausible networks.

\subsection{From prediction to global patterns}

Although prediction is a powerful tool in the immediate/local sense 
(\emph{e.g.,} it can allow local land managers/custodians to have a 
first approximation
of how species may be interacting in that given area it is of course
also a possible what to fill in the global map. A 'filled map' will
allow us to be poised to develop a more mechanistic, global scaled,
understanding of networks. The work presented in this thesis represents
but a tentative and small initial foray in this direction. Namely
Chapter~\ref{SVD}, which presents a different, more information theory 
approach to defining complexity using the singular value vector component
of an SVD \cite{Shannon1948MatThe}. The final chapter of this thesis
(Chapter~\ref{SpatialBoundaries}) is a \texttt{Julia} package that
allows users to implement the Wombling algorithm (an edge
detection mechanism; \cite{Womble1951DifSys}).

Wombling has been discussed as a useful tool for spatial analyses in
ecology \cite{Fortin2005SpaAna} and has been used to detect transitions
across a landscape \cite{Philibert2008SpaStr}, changes in biological
variables in communities \cite{Barbujani1989DetReg} and to analyse the
spread of invasive species \cite{Fitzpatrick2010EcoBou}. Being able to
subdivide networks into patches within a landscape will help us better
understand the boundaries of and between networks as well as how these
may relate to species or community changes and boundaries - such as when
transitioning across habitat `boundaries' \cite{Hackett2019ResOur}.

\subsection{Objectives}\label{objectives-and-hypotheses}

Being able to understand, quantify, and work with ecological networks is
important from a conservation and land management perspective as this
will have cascading implications with regards to ecosystem functioning
and stability. Yet we are severely hamstrung by a lack of high-quality,
usable data as well as an appropriate set of tools that can be used to
contextualise and understand ecological networks. There is a need for
tools that can help us construct networks for where there are no data
\emph{i.e.,} make predictions as well as developing tools (or ideas) that
can be used to help further our mechanistic understanding of networks once
we are at a point where we have the large scale data to do so. My work will
help address these two issues in the context of developing tools that will
either directly enable us to make predictions (Chapters~\ref{Roadmap},
\ref{Perspectives}, and \ref{Foodweb}), or present methods that are aligned
with global (large-scale) questions that will allow us to compare networks
(Chapter~\ref{SVD}) or attempt to delineate them 
(Chapter~\ref{SpatialBoundaries}).

\section{Methodological overview}

\subsection{Contemplation and consideration}

In the following sections I will break these ideas and questions down
further, focusing on how we can access (decode) the information in
networks, and how this can be used to inform us on the amount of
information contained in networks as well as as the applicability within
the context of network prediction - with the specific aim of
constructing the Canadian metaweb. Once we have the ability to predict
networks we are then poised to begin approaching detecting boundaries
\emph{i.e.} changes between networks across space. Broadly this will
address two areas of which we currently have a gap within network
ecology \emph{i}) overcoming the issue of lack of empirical data of
networks \cite{Poisot2021GloKno} through prediction and \emph{ii})
beginning an initial foray into understanding the variation of networks
over space - another aspect of network ecology that is still in its
infancy (although see \cite{Fortin2021NetEcoa} for a discussion of the
possibilities of this direction of study).

\subsection{Transfer learning for network prediction}

Transfer learning is a machine learning methodology that uses the 
knowledge gained when solving a known problem and
applying it to solve a (related) problem by transferring the knowledge
across a shared medium (space) \cite{Torrey2010TraLea, Pan2010SurTra}.
The concept of transfer learning is
an approach that is particularly well suited for the problem of network
prediction as it allows us to lean on the data that are available to 
enable us to make \emph{de novo} predictions. This could be as simple
as pinpointing missing interactions in the existing data 
(\emph{e.g.,} pairwise learning has been used to predict plant-pollinator
interactions in \cite{Stock2021PaiLea}) as well as a way to predict
novel interactions (\emph{i.e.,} fill in those global gaps). This,
in a sense allows us to bring knowledge with us from an area for
which we \emph{have} data to an area where it is \emph{lacking}. In the
case of predicting species interactions transfer learning as an idea 
works because interactions are being conserved at an evolutionary scale using
phylogenetic relatedness for a given community can inform as how they
interact with each other \cite{Davies2021EcoRed, Elmasri2020HieBay,
Gomez2010EcoInt}. Chapter~\ref{Foodweb} presents a transfer learning framework
and uses the task of constructing a Canadian metaweb using the European metaweb
assembled by \cite{Maiorano2020TetSpe} as a proof-of-concept the workflow and
data itself can be found \href{https://osf.io/2zwqm/}{here}. Below is a
high-level summary of that framework.

\subsubsection{Learning using graph embedding}

Before one can transfer any knowledge we must first learn something about
the system using known data. Since ecological networks can be represented
by their adjacency matrices we can turn to graph theory to help us find a
way to learn something about the known interaction network. Graph embedding
is a low dimensional representation of the graph but, importantly, still
preserves its topology \cite{Yan2005Graph}. This process essentially allows
as to learn something about where species (nodes) are situated within the
network - which (in an abstract way) informs us of the the role a species
plays in the community. There are multiple embedding approaches discussed
in Chapter~\ref{Perspectives}, but in the context of the framework developed
in Chapter~\ref{Foodweb} we will focus on the use of SVD as an embedding 
technique. SVD presents an appropriate embedding of ecological networks, having
been shown to both capture their complex, emerging properties
\cite{Strydom2021SvdEnt} and allow for the highly accurate prediction
of the interactions within a single network \cite{Poisot2021ImpMam}.

\subsubsection{Graph embedding using SVD}

Singular Value Decomposition \cite{Forsythe1967ComSol, Golub1971SinVal} 
is the factorisation of an adjacency matrix \(\mathbf{A}\) (where
\(\mathbf{A}_{m,n} \in\mathbb{B}\)) into the form:

\[ \mathbf{U}\cdot\mathbf{\Sigma}\cdot\mathbf{V}^T \]

Where \(\mathbf{U}\) is an \(m \times m\) orthogonal matrix and
\(\mathbf{V}\) an \(n \times n\) orthogonal matrix. The columns in these
matrices are, respectively, the left- and right-singular vectors of
\(\mathbf{A}\). \(\mathbf{\Sigma}\) is a diagonal matrix that contains
only non-negative \(\sigma\) values. Where \(\sigma_{i} = \Sigma{ii}\),
and contains the singular values of \(\mathbf{A}\).
 
An SVD can be truncated so as to remove additional noise in the dataset by
omitting non-zero and/or smaller \(\sigma\) values from
\(\mathbf{\Sigma}\) using the rank of the matrix. Under a t-SVD
\(\mathbf{A}_{m,n}\) is decomposed so that \(\mathbf{\Sigma}\) is a
square \(r \times r\) diagonal matrix (whith \(1 \le r \le r_{full}\)
where \(r_{full}\) is the full rank of \(\mathbf{A}\) and \(r\) the rank
at which we truncate the matrix). Additionally, \(\mathbf{U}_{t}\) is now a
\(m \times r\) semi unitary matrix and \(\mathbf{V}'_{t}\) a \(n \times r\)
semi-unitary matrix.

In the context of 'learning using embedding' the learned information is
captured using an SVD, however for the task of network prediction we modified
the products of the SVD so that they could be used for an RDPG. An RDPG 
estimates the probability of observing interactions between
nodes (species) as a function of the nodes' latent variables, and is a
way to turn an SVD (which decompose one matrix into three) into two
matrices that can be multiplied to provide an approximation of the
network. The latent variables used for the RDPG, called the left and
right subspaces, are defined as
$\mathscr{L} = \mathbf{U}\sqrt{\mathbf{\Sigma}}$, and
$\mathscr{R} = \sqrt{\mathbf{\Sigma}}\mathbf{V}'$ -- using the full
rank of $\mathbf{A}, \mathscr{L}\mathscr{R} = \mathbf{A}$, and
using any smaller rank results in
$\mathscr{L}\mathscr{R} \approx \mathbf{A}$. 
(hereafter referred to the left ($\hat{\mathscr{L}}$) and right 
($\hat{\mathscr{R}}$) subspaces respectively). These
subspaces are ecologically informative and tell us about the 'generality'
(think predator capacity, \emph{sensu} \cite{Schoener1989Food}) and 
'vulnerability' (think capacity to be prey, \emph{sensu} \cite{Schoener1989Food})
of the species in the European network. This in essence provides us with
an idea of where a species is likely to occur within a network/the space
it occupies in the network.

\subsubsection{Transferring and inferring using phylogenetic relatedness}

In order to transfer the knowledge (the generality and vulnerability
values) from the European metaweb to the
Canadian species pool, we performed ancestral character estimation using
a Brownian motion model and the Upham \cite{Upham2019InfMam} mammalian
phylogeny. This uses the estimated
feature vectors (left and right subspaces) for the European mammals to
create a state reconstruction for all species and allows us to impute
the missing generality and vulnerability values for the Canadian
species that are not already in the European network. Essentially this
allows us to infer where in the two subspaces the Canadian mammals are
located.

\subsubsection{Novel Prediction using RDPG}

AS we now essentially have the left and right subspaces for the Canadian
metaweb we can directly multiply these to yield the metaweb, specifically
using an RDPG. The phylogenetic reconstruction of ($\hat{\mathscr{L}}$) and
($\hat{\mathscr{R}}$) has an associated uncertainty, therefore, we can
assemble a \emph{probabilistic} metaweb in \emph{sensu} \cite{Poisot2016StrPro},
\emph{i.e.} in which every interaction is represented as a single,
independent, Bernoulli event of probability \(p\).

\subsection{SVD entropy: a measure of network
complexity}\label{svd-entropy-a-measure-of-network-complexity}

Singular value decomposition also offers two interesting candidate measures
of complexity for ecological networks. The first measure is the rank of
the matrix. The rank of $\mathbf{A}$ (noted as
$r = \text{rk}(\mathbf{A})$) is the dimension of the vector space
spanned by the matrix and corresponds to the number of linearly
independent rows or columns, which works as an estimate of `external
complexity', in that it describes the dimension of the vector space of
the matrix, and therefore the number of linearly independent rows (or
columns) of it. From an ecological standpoint, this quantifies the
number of unique `strategies' represented in the network.

The second measure is to calculate the entropy of the matrix obtained
through SVD by using the singular values \cite{@Shannon1948MatThe}. This
so-called SVD entropy measures the extent to which each rank encodes an
equal amount of information, as the singular values capture the
importance of each rank to reconstruct the original matrix; this
approach therefore serves as a measure of `internal complexity'.

Intuitively, the singular value $i$ ($\sigma_i$) measures how much
of the dataset is (proportionally) explained by each vector - therefore,
one can measure the entropy of \(\mathbf{\sigma}\) following
\cite{Shannon1948MatThe}. High values of SVD entropy reflects that all vectors
are equally important, \emph{i.e.} that the structure of the ecological
network cannot efficiently be compressed, and therefore indicates high
complexity \cite{Gu2016HowLon}. Because networks have different
dimensions, we can use Pielou's evenness \cite{Pielou1975EcoDiv} to
ensure that values are lower than unity, and quantify SVD entropy, using
$s_i = \sigma_i/\text{sum}(\sigma)$ as:

$$J = -\frac{1}{\ln(k)}\sum_{i=1}^k s_i\cdot\ln(s_i)$$

Where \(k = \text{rk}(\mathbf{A})\) \emph{i.e.} the rank of the matrix,
which is equal to the number of non-zero entries in \(\mathbf{\Sigma}\)
as per the Eckart-Young-Mirsky theorem \cite{Eckart1936AppOne,
Golub1987GenEck}. 

We used all bipartite networks contained in the \texttt{web-of-life.es}
database and measured both their external (rank) and internal (SVD entropy)
complexity. As this database extracted species interaction networks from
supplementary materials across all inhabited continents and covers a large range
of sampling years, environments, organisms, and sampling methodologies. As such,
this dataset is particularly suited to describe the general trends across
\emph{global} ecological networks.

\subsection{Spatial wombling for edge detection}

Spatial wombling (an edge detection algorithm; \cite{Womble1951DifSys}) could
prove to be a promising approach to trying to detect boundaries between
networks. Chapter~\ref{SpatialBoundaries} presents a \texttt{Julia} package that
implements both the lattice and triangulation wombling algorithms. 

Broadly, wombling interpolates between a given set of points.
Two core metrics that are derived by this process is the rate of change
\(m\) which is calculated as:

$$m = \sqrt{\frac{\partial f(x,y)}{\partial x}^2 + \frac{\partial
f(x,y)}{\partial y}^2}$$

This can be used to find the zones (or cells) of rapid change across the
landscape and identify potential candidate boundaries. It is also
possible to calculate the direction (\(\theta\)) for each rate of
change. This is calculated as:

$$\theta = \arctan \left( \frac{\partial f(x,y)}{\partial y} \bigg/ \frac{\partial f(x,y)}{\partial x} \right) + \Delta$$

$$\text{where} \quad \Delta =
\left\{ \begin{array}{ccc}
    0 \degree & \text{if} & \frac{\partial f(x,y)}{\partial x} \geq 0 \\
    180 \degree & \text{if} & \frac{\partial f(x,y)}{\partial x} < 0 \\
\end{array} \right\}$$

Both $m$ and $\theta$ are an approximation on the `topology' of a
certain metric ($z$, \emph{e.g.} number of species) between a
collection of points. Similarity between the $z$ values indicates a
uniformity between those points and thus a low rate of change where as a
high degree of difference for the given window is indicative of rapid
change \emph{i.e.} a boundary as we transition from one zone to the
next.

\subsubsection{Lattice wombling}

For a lattice of points where one will have sampling locations arranged
uniformly across space one can simply use the co-ordinates 'as is' and 
$f(x,y)$ can be calculated as:

$$f(x,y) = z_{1}(1-x)(1-y) + z_{2}x(1-y) + z_{3}x y + z_{4}(1-x)y$$

For convenience the centroid of the square \emph{i.e.} the values of
$x$ and $y$ can be standardised to 0.5. 

\subsubsection{Triangulation wombling}

When working with points that are irregularly distributed across the 
landscape it is possible to use triangulation wombling 
\cite{Fortin1995DelEco}. The three nearest
neighbours can be determined using a Delaunay triangulation algorithm
\cite{Delaunay1934SphVid} and $f(x,y)$ can be calculated as:

$$f(x,y) = ax + by + c$$

where:

$$ \left[ \begin{array}{ccc} a & b & c \end{array} \right] = 
\left[ {\begin{array}{ccc}
   x_{1} & y_{1} & 1\\
   x_{2} & y_{2} & 1\\
   x_{3} & y_{3} & 1\\
  \end{array} } \right]^{-1}\cdot
  \left[
  \begin{array}{ccc} z_{1} & z_{2} & z_{3} \end{array} \right]$$

and the position of the centroid being calculated as:

$$ \Big( \frac{x_{1} + x_{2} + x_{3}}{3} \Big), \Big( \frac{y_{1} + y_{2} +
y_{3}}{3} \Big) $$

\subsubsection{Boundary detection}

Detecting boundaries \emph{i.e.} areas where the angle of the landscape
transitions sharply is surprisingly simple. After having calculated the
rate of change (\(m\)) for the geographical area it is possible to use
these values to identify and assign potential boundaries
\cite{Fortin2005SpaAna, Oden1993CatWom, Fortin1995DelEco}. Following the
approach outlined in \cite{Fortin2005SpaAna} a threshold value (or
percentile class) can be set and will determine what proportion of cells
will be retained as potential boundaries. For example if using a 0.1
threshold then the highest 10\% of points (which are ranked based on 
\(m\)) will be classified as candidate boundaries.

\section{Chapter summaries}

\subsection{Chapter 1: A roadmap for predicting ecological networks}

\cite{Strydom2021RoaPre} (as the name suggests) maps out a series of questions
and considerations with regards to approaching the challenge of
predicting species interactions across space and time. Points of
discussion regarding predicting network structure and interactions
include the value of the simultaneous prediction of occurrence and
interactions (\emph{e.g.} \cite{Zurell2020TesSpe}, why connectance
as a useful property to use for predictions and dealing with 
potentially novel interactions.

\subsection{Chapter 2: Graph embedding for network prediction}

This chapter should be viewed as a companion piece for Chapter~\ref{Foodweb}
as it provides larger conceptual discussions around the potential
applications of graph embedding and how these relate to transfer
learning. In addition to the more methodological discussions we
also delve into the applications of prediction and how this should
be done in a way that does 'no harm'. This section also includes
what could be considered some vital discussions on how we can
advance the way we think about and define the original definition
of a 'metaweb' \emph{sensu} \cite{Dunne2006NetStr}.

\subsection{Chapter 3: Prediction in action: The Canadian
Metaweb}

Building on the ideas in \cite{Strydom2021RoaPre}, work on the use of 
transfer learning for predicting \emph{de novo} interactions
\cite{Runghen2021ExpNod}, and the applicability of phylogenetic
reconstruction within the context of ecological networks \emph{e.g.,}
\cite{Braga2021PhyRec}, we set out to create a probabilistic metaweb for
terrestrial Canadian mammals in \cite{Strydom2021FooWeb}. What is 
perhaps most exciting about this chapter is that despite sharing about
only 4\% of species between Canada and Europe
we were able to construct a metaweb that correctly predicted about 91\%
of the species interactions in Canada (this was achieved by using known
datasets to validate predicted interactions), which speaks to the
`robustness' of this transfer learning framework and potential
applicability in a variety of settings (\emph{e.g.,} as an `informative prior'
from which more localised/spatially explicit networks can be constructed
\cite{Cirtwill2019QuaFra}) to help in filling in the blank
spaces on the map when it comes to regional food webs.

\subsection{Chapter 4: SVD entropy: a measure of network complexity}

In \cite{Strydom2021SvdEnt} we present SVD entropy as a starting point to
unifying (and standardising) how we define the complexity of ecological
networks. We argue that having a unified definition will allow us to
revisit how complexity relates to the ecological properties of networks
using a standardised framework that focuses on the physical complexity
of networks as opposed to the complexity of the behaviour of the system.
An interesting result from using SVD entropy is that the complexity of
ecological networks is indeed \emph{immense}, yet despite this high
complexity networks are still not reaching their \emph{maximum}
potential complexity, which raises questions for future contemplation
with regards to understanding what might be constraining network
complexity. An additional observation is that the relationship between 
network size, connectance and complexity. Results
point to the potential constraint of network size on complexity.
Networks at the early assembly stages tend to be severely constrained
\cite{Barbier2018GenAss, Saravia2018EcoNet} due to conditions needed
for the persistence of multiple species. As networks grow larger, these
constraints may ``relax'', leading in networks with more redundancy, and
therefore a lower complexity.

\subsection{Chapter 5: SpatialBoundaries.jl: a software for boundary detection}

Here we present a \texttt{Julia} package, (the documentation is available
\href{https://poisotlab.github.io/SpatialBoundaries.jl/dev/}{here})
that has the functionality to implement the spatial wombling algorithm
across both a uniform landscape \emph{i.e.} lattice wombling as well as
irregular/random landscapes \emph{i.e.} triangulation wombling. These
two methods still calculate the rate of change (\(m\)) and
directionality (\(\theta\)) in the same manner but differ in how the
aggregate and quantify the surface for a set of points
\cite{Fortin2005SpaAna}. These are the two algorithms implemented in the
\texttt{SpatialBoundaries.jl} package, and should provide functionality
for most use cases when data are quantitative. \texttt{SpatialBoundaries.jl} 
has also been developed so as to integrate with other packages such as
\texttt{SimpleSDMLayers.jl}.

\section{Conclusion}\label{conclusion}

As a whole this thesis should be viewed as a computational toolbox for
network ecology that addresses both the issue of data scarcity through
the use of predictive tools (addressing the `Eltonian shortfall' highlighted by \cite{Hortal2015SevSho}) as well as presenting methods/ideas geared towards thinking about networks at global scales. This means that we would \emph{i}) have `tangible' networks from which we can begin to work with in various contexts or situations and \emph{ii}) have new methods/tools to begin asking questions about netwroks at a global scale. In other words adding more building blocks from which we can begin to take network ecology to the next level, \emph{i.e.,} bridging the gap from 'local-level network understanding' to 'tools for global netwrok analysis'.

\bibliographystyle{plain}
\sectionbibliography{ref_Intro.bib}

\endinput
%%
%% End of file `introduction.tex'.


%%%%%%%%%%%%%%%%%%%%%%%%%%%%%%%%%%%%%%%%%%%%%%%%%%%%%%%%%%%%
%%%%%%%%%%%%%%%%%%%%                   %%%%%%%%%%%%%%%%%%%%%
%%%%%%%%%%%%%%%%%%%%  A R T I C L E S  %%%%%%%%%%%%%%%%%%%%%
%%%%%%%%%%%%%%%%%%%%                   %%%%%%%%%%%%%%%%%%%%%
%%%%%%%%%%%%%%%%%%%%%%%%%%%%%%%%%%%%%%%%%%%%%%%%%%%%%%%%%%%%

\include{02_Roadmap}
\include{04_Perspectives}
\include{03_Foodweb}
\include{01_SVD}
\include{05_SpatialBoundaries}
\anglais
\chapter{General Conclusion}\label{Conclusion}

As species interaction networks are determined by ecological and
evolutionary mechanisms that have played out across spatial and temporal
scales the measures that define their structure also capture the
processes that have played a role in structuring them. Thus the
properties of a network are not only representative of its structure but
also of \emph{how} different processes have played a role in determining
it \emph{i.e.,} there is an element of predictability in being able to
quantify the structure of a network. For example \cite{MacDonald2020RevLin}
and \cite{Banville2023What} show that, after taking into account biological constraints, it is possible to make inferences as to the properties of networks. The ability to use a 'simple' measures of the community for a given location and to have an estimate of the structure of the potential network is truly
amazing. To me this is echoed in Chapter~\ref{Foodweb} and really showcases that
with very little 'real world' information we can still make really solid
predictions due to the 'information' encoded in networks.

\section{Scrutinising our methods}

Something something that the job isn't done when it comes to really looking at the data we are using for prediction. More recent work is showing that the imbalances in current data might be a bigger problem than what we would like to admit (especially the false negative rate, \emph{i.e.,} interactions that do occur but are missed in the field). When reading the work from \cite{Poisot2023Guidelines} and \cite{Catchen2023Missing} one can't help but get a bit hesitant to jump on the prediction bandwagon, however I console myself that we are able to show in \autoref{rdpg-yields-an-accurate-classifier} that the transfer learning model does do quite well even when we bring false interactions into the dataset. Of course this does highlight the need to be critical (or at least cautious) when it comes to using datasets for learning, and highlights the need for identifying priority sampling locations and (maybe) even priority interactions, \emph{e.g.,} \href{https://github.com/EcoJulia/BiodiversityObservationNetworks.jl/tree/main}{some} of the work coming out of the GeoBon group
focusing on locating priority sampling locations 

Something about how we can try and push the limits of the reconstruction
chapter further. Some ideas include maybe looking at rewilding projects
and seeing if we are able to recover those interactions (or maybe even 
invasion studies). There is also the possibility for looking at 'distant
relatives' communities e.g., trying to recover Australian interactions.
But the idea of thinking about construction in the context of rewilding
is extremely tempting...

\section{Defining ecotrophic zones}

Networks are dynamic, and they can vary across space
\cite{Golubski2016EcoNet, Vazquez2007SpeAbu} or time
\cite{Poisot2015SpeWhy, Trojelsgaard2016EcoNet} as a function of the
environmental conditions. Naturally, we expect network properties to
also be dynamic and vary over - in this instance - space. Spatial
wombling can be used as a starting for understanding \emph{how} networks
vary in across a landscape. An initial direction to push this idea is . 

This comparison to species turnover
lends itself to an interesting comparison - do we see similar patterns
of rates of change at the species or community level and with regards to
network structure? For example although we might expect species
composition to change along a gradient it may be that they are replaced
by a functional equivalent and we may not expect to see rapid change or
turnover with regards to network structure \emph{sensu lato} conserved
backbones \cite{Mora2018IdeCom}. Alternatively, intraspecific variation
could drive interaction turnovers even without changes in species
composition \cite{Bolnick2011WhyInt} \emph{i.e.,} a re-wiring of
interactions. Indeed, \cite{Martins2022Global} showcases this with avian
frugivory networks where species-level differences are linked to environmental
changes but not network structure.

\subsection{How does the within structure of networks vary?}

There is also the scope to develop a more nuanced understanding of how
the landscape structures networks. Specifically how the different nodes
(\emph{i.e.} species) of the network will perceive the landscape
differently. Which means that we might expect \emph{within} network
changes \emph{i.e.} change in motifs/graphlets across the landscape, although deciding exactly what measure might actually be driving differences between local networks and the regional metaweb might not be that simple \cite{Saravia2022Ecological}. That being said \cite{Fortin2021NetEcoa} provide a compelling argument for the need to `combine' these smaller functional units with larger spatial networks (and arguably even thinking about the interplay of time and space \cite{Estay2023Editorial}). \autoref{supp:boundaries} shows some initial (and by no means well resolved) ideas of how we can use the \texttt{SpatialBoundaries.jl} along with the metacommunity model developed by \cite{Thompson2017Dispersal} to look at the how the environmental, species community, and network communities compare within a landscape.

\subsection{Boundaries for policy or management}

Although this section argues for a more theoretical approach to understanding boundaries in the context of understanding potential assembly patterns/constraints there is also a strong argument for being able to draw lines around communities (or (realistically) provide an argument to challenge it). In \autoref{the-metaweb-merges-ecological-hypotheses-and-practices} and \autoref{conclusion-metawebs-predictions-and-people} we briefly mention that the scale of prediction should be relevant, but should also take into consideration people. To me there is an argument that this is also the case when thinking about network boundaries. Unfortunately policy and legislation are enacted at various levels of government/ruling bodies. Being able to \emph{show} the boundaries between networks may in fact be a powerful tool at the governance level as it is surely more meaningful than looking at the species community. Although I feel it is important to stress that the idea of trying to draw boundaries should be approached with caution and sensitivity, and to me echos many of the sentiments discussed in Box 2 of \autoref{box:people}.

\section{The future collaborative toolbox}

On a probably more contemplative closing note I want to discuss the value of thinking about the development of further tools for the toolbox analogy used in this thesis. A non-insignificant amount of work in this thesis was only possible with the support and intellectual contribution of many collaborators and there is an argument for continuing this strong network of collaboration for future tools. From a purely practical perspective the continued push for developing biology-centric packages within the \texttt{Julia} language \cite{Roesch2021Julia} requires that we maintain interoperability between packages and build a collection that build on and fit in amongst each other. Looking at the science/theoretical side of the toolbox, a unified idea or goal for moving the macroecological network 'agenda' means that we can build on ideas and thoughts to a more collective research agenda. For example \cite{Dansereau2023Spatially, Catchen2023Improving, Banville2023What} have all already used the work presented in this thesis to take the ideas discussed in new and further directions. And although this is not to say that we should not also work on developing 'competing' methods (although I would argue 'competing' here it is used in the context for finding alternative approaches to solving a similar problem \emph{e.g.,} \cite{Caron2022Addressing} takes a more trait-based approach to network prediction) there is strong evidence that in working together we can get where we want to be sooner. 

\bibliographystyle{plain}
\sectionbibliography{ref_Intro.bib}

\endinput

%%
%% etc.

 % S'il y a une bibliographie pour tout le document, on peut
 % utiliser les commandes suivantes. À noter que le style est
 % laisser au choix de l'auteur·e. (Il est même possible
 % d'utiliser <natbib>).
 % Il est possible d'avoir une bibliographie pour chaque
 % chapitre. Consulter l'article en exemple pour voir
 % comment faire.
%%\bibliographystyle{plain-fr}
%%\bibliography{<fichier.bib>}

 % Pour les annexes :
\appendix
 % Les annexes se font comme les chapitres. Le fichier
 % commence par \francais ou \anglais et ensuite
 % \chapter{..}. Le reste est parreil à un chapitre normal.
\include{S_foodweb}
\anglais
\chapter{Understanding where networks stop}\label{supp:boundaries}

\section{Why boundaries are interesting}

As discussed in both Chapters~\ref{Perspectives} and \ref{SpatialBoundaries}
there is value in thinking about the existence of boundaries between networks,
either from a prediction perspective (\emph{e.g.,} knowing at what scale to 
make predictions at) or from a more theoretical question of where do networks
stop?

\section{A metacommunity model for boundary detection}

The metacommunity model developed by \cite{Thompson2017Dispersal} is a good
starting point to use for \cite{Thompson2017Dispersal} is tritrophic food 
webs (so 'plants', 'herbivores', and 'carnivores') and provides 
an ideal 'toy environment' with which to 'play' with. The model
(\ref{eq:metacomm_full}) itself is a collection of modified Lotka–Volterra
equations and (broadly) models species abundance as a function of interaction
strength, environmental effect, immigration, and emigration. The metacommunity consists of $S$ species with $M$ environmental patches and looks as follows:

\begin{equation} \label{eq:metacomm_full}
X_{ij}(t+1)=X_{ij}(t)exp\left[C_{i} + \sum_{k=1}^{S}B_{ik}X_{kj}(t)+A_{ij}(t)\right]+I_{ij}(t)-X_{ij}(t)a_{i}
\end{equation}

Where $X_{ij}(t)$ is the abundance of species $i$ in patch $j$ at time $t$. $C_i$
is its intrinsic rate of increase (which we have set to 0.1 for 'plants' and
-0.01 for 'herbivores' and 'carnivores'). $B_{ik}$ is the per capita effect of
species $k$ on species $i$. The exact interaction strength for each species pair
is drawn from a uniform distribution with the parameters for the 
interaction pairs listed in \autoref{table:interaction_strength}, the values
drawn from the uniform distribution are scaled by dividing by $0.33S$ to yield
the final interaction strength for each interacting pair.

\begin{table}[h!]
\centering
\begin{tabular}{||c c||} 
 \hline
Interacting pair & Range of uniform distribution \\ [0.5ex] \hline\hline
 Plant-plant & -1 -- 0 \\ 
 Plant-herbivore & 0 -- 0.1 \\
 Plant-carnivore & 0 \\
 Herbivore-plant & -0.3 -- 0 \\
 Herbivore-herbivore & -0.2-- -0.15 \\
 Herbivore-carnivore & 0 -- 0.08 \\
 Carnivore-plant & 0  \\
 Carnivore-herbivore & -0.1 -- 0  \\
 Carnivore-carnivore & -0.2-- -0.15 \\ [1ex] 
 \hline
\end{tabular}
\caption{Intervals used for the uniform distribution from which interaction
strengths values are drawn from for the different types of species pair
interactions. Note this is represent the effect of species type 1 on species
type 2 \emph{i.e.,} herbivore-plant represents the effect of a herbivore species
on a plant species}
\label{table:interaction_strength}
\end{table}

$A_{ij}(t)$ is the effect of the environment in patch $j$ on species $i$ at time $t$ and can be further expanded as follows:  

\begin{equation} \label{eq:metacomm_env}
A_{ij}(t)=h\left(exp-\frac{(E_{j}(t)-H_{i})^2}{2\sigma^2}-1\right)
\end{equation}

Species environmental optima ($H_i$) are evenly distributed across the entire
range of environmental conditions for each trophic level, meaning that species
from different trophic levels will be at, or near the same environmental optima. 
$h$ is a scaling parameter (set to 300), $E_j(t)$ is the environment in patch
$j$ at time $t$ and $\sigma$ is the standard deviation (set to 50).

$I_{ij}(t) $is the abundance of species $i$ immigrating to patch $j$ at time $t$
and can be expanded as follows:

\begin{equation} \label{eq:metacomm_imm}
I_{ij}(t)=\sum_{l=j}^{M}a_iX_{il}(t)exp(-Ld_{jl})
\end{equation}

Where $ai$ is the proportion of the population of species $i$ that disperses at
each time step, the dispersal rate is drawn from a normal distribution ($\mu$ =
0.1, $\sigma$ = 0.025) for each species. The abundance of immigrants to patch
$j$ from all other patches is governed by where $d_{jl}$ is the geographic
distance between patches $j$ and $l$, and $L$ (the strength of the exponential
decrease in dispersal with distance), which is also drawn from a normal
distribution for each species. The parameters used for $L$ are trophic level
dependant and are show in \autoref{table:interaction_decay}

\begin{table}[h!]
\centering
\begin{tabular}{||c c c||} 
 \hline
Trophic level & $\mu$ & $\sigma$ \\ [0.5ex] \hline\hline
 Plant & 0.3 & 0.075 \\ 
 Herbivore & 0.2 & 0.05 \\
 Carnivore & 0.1 & 0.025 \\ [1ex] 
 \hline
\end{tabular}
\caption{Parameters for the normal distributions used to determine the dispersal
decay ($L$) for each species depending on its trophic level.}
\label{table:interaction_decay}
\end{table}

For the initial modelling exercise presented below we had 80 species ($S$) and
a 200 by 200 landscape ($M$). The associated code for this mode can be found in a
GitHub repo \href{https://github.com/PoisotLab/Omnomnomivores}{here}.
%%
%% etc.

\end{document}

\endinput
%%
%% End of file `gabaritTPA.tex'.
