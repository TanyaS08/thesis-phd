\anglais
\chapter{Understanding where networks stop}\label{supp:boundaries}

\section{Why boundaries are interesting}

As discussed in both Chapters~\ref{Perspectives} and \ref{SpatialBoundaries}
there is value in thinking about the existence of boundaries between networks,
either from a prediction perspective (\emph{e.g.,} knowing at what scale to 
make predictions at) or from a more theoretical question of where do networks
stop?

\section{A metacommunity model for boundary detection}

The metacommunity model developed by \cite{Thompson2017Dispersal} is a good
starting point to use for \cite{Thompson2017Dispersal} is tritrophic food 
webs (so 'plants', 'herbivores', and 'carnivores') and provides 
an ideal 'toy environment' with which to 'play' with. The model
(\ref{eq:metacomm_full}) itself is a collection of modified Lotka–Volterra
equations and (broadly) models species abundance as a function of interaction
strength, environmental effect, immigration, and emigration. The metacommunity consists of $S$ species with $M$ environmental patches and looks as follows:

\begin{equation} \label{eq:metacomm_full}
X_{ij}(t+1)=X_{ij}(t)exp\left[C_{i} + \sum_{k=1}^{S}B_{ik}X_{kj}(t)+A_{ij}(t)\right]+I_{ij}(t)-X_{ij}(t)a_{i}
\end{equation}

Where $X_{ij}(t)$ is the abundance of species $i$ in patch $j$ at time $t$. $C_i$
is its intrinsic rate of increase (which we have set to 0.1 for 'plants' and
-0.01 for 'herbivores' and 'carnivores'). $B_{ik}$ is the per capita effect of
species $k$ on species $i$. The exact interaction strength for each species pair
is drawn from a uniform distribution with the parameters for the 
interaction pairs listed in \autoref{table:interaction_strength}, the values
drawn from the uniform distribution are scaled by dividing by $0.33S$ to yield
the final interaction strength for each interacting pair.

\begin{table}[h!]
\centering
\begin{tabular}{||c c||} 
 \hline
Interacting pair & Range of uniform distribution \\ [0.5ex] \hline\hline
 Plant-plant & -1 -- 0 \\ 
 Plant-herbivore & 0 -- 0.1 \\
 Plant-carnivore & 0 \\
 Herbivore-plant & -0.3 -- 0 \\
 Herbivore-herbivore & -0.2-- -0.15 \\
 Herbivore-carnivore & 0 -- 0.08 \\
 Carnivore-plant & 0  \\
 Carnivore-herbivore & -0.1 -- 0  \\
 Carnivore-carnivore & -0.2-- -0.15 \\ [1ex] 
 \hline
\end{tabular}
\caption{Intervals used for the uniform distribution from which interaction
strengths values are drawn from for the different types of species pair
interactions. Note this is represent the effect of species type 1 on species
type 2 \emph{i.e.,} herbivore-plant represents the effect of a herbivore species
on a plant species}
\label{table:interaction_strength}
\end{table}

$A_{ij}(t)$ is the effect of the environment in patch $j$ on species $i$ at time $t$ and can be further expanded as follows:  

\begin{equation} \label{eq:metacomm_env}
A_{ij}(t)=h\left(exp-\frac{(E_{j}(t)-H_{i})^2}{2\sigma^2}-1\right)
\end{equation}

Species environmental optima ($H_i$) are evenly distributed across the entire
range of environmental conditions for each trophic level, meaning that species
from different trophic levels will be at, or near the same environmental optima. 
$h$ is a scaling parameter (set to 300), $E_j(t)$ is the environment in patch
$j$ at time $t$ and $\sigma$ is the standard deviation (set to 50).

$I_{ij}(t) $is the abundance of species $i$ immigrating to patch $j$ at time $t$
and can be expanded as follows:

\begin{equation} \label{eq:metacomm_imm}
I_{ij}(t)=\sum_{l=j}^{M}a_iX_{il}(t)exp(-Ld_{jl})
\end{equation}

Where $ai$ is the proportion of the population of species $i$ that disperses at
each time step, the dispersal rate is drawn from a normal distribution ($\mu$ =
0.1, $\sigma$ = 0.025) for each species. The abundance of immigrants to patch
$j$ from all other patches is governed by where $d_{jl}$ is the geographic
distance between patches $j$ and $l$, and $L$ (the strength of the exponential
decrease in dispersal with distance), which is also drawn from a normal
distribution for each species. The parameters used for $L$ are trophic level
dependant and are show in \autoref{table:interaction_decay}

\begin{table}[h!]
\centering
\begin{tabular}{||c c c||} 
 \hline
Trophic level & $\mu$ & $\sigma$ \\ [0.5ex] \hline\hline
 Plant & 0.3 & 0.075 \\ 
 Herbivore & 0.2 & 0.05 \\
 Carnivore & 0.1 & 0.025 \\ [1ex] 
 \hline
\end{tabular}
\caption{Parameters for the normal distributions used to determine the dispersal
decay ($L$) for each species depending on its trophic level.}
\label{table:interaction_decay}
\end{table}

For the initial modelling exercise presented below we had 80 species ($S$) and
a 200 by 200 landscape ($M$). The associated code for this mode can be found in a
GitHub repo \href{https://github.com/PoisotLab/Omnomnomivores}{here}.