%% To change chapter header dynamically from french to english, use
%%\entetedynamique
\setcounter{corA}{0} % Pour recommencer à compter les def,
                     % theo, etc. à partir de 1
\setcounter{secnumdepth}{1}
\renewcommand{\thefigure}{\arabic{figure}}
\setcounter{figure}{0}
 % Pour écrire un article en français
% \francais
 % Pour écrire un article en anglais
\anglais
%% NOTE: La plupart des macros ont un nom en anglais.
%% P.ex. \adresse et \address fonctionnent et sont équivalents.
%% \revue=\journal
%% \auteur=\author
%% \titre=\title

 % Nom de la revue de publication
\revue{Global Ecology and Biogeography}
\article{Evaluating ecological uniqueness over broad spatial extents using species distribution modelling}

 % Contribution(s) personnelle(s) à l'article et rôle joué par tous les coauteurs
 %
 % Nécessaire seulement lorsque vous n'êtes pas seul·e auteur·e.
 % Les contributions peuvent apparaître ailleurs dans la thèse.
 % Si \contributions est laissé vide (p.ex. si vous effacez
 % celui ci-bas), aucune contributions ne seront générées sur
 % la page titre de l'article.
 %
 % REMARQUE : L'exemple est sous forme d'\itemize,
 % mais toutes les constructions \LaTeX\ sont permises.
 % La commande peut aussi contenir un simple petit texte.
 %
 % La commande admet une option [<entête>]
\contributions[]%[Mes contributions et le rôle des coauteurs]
{
    GD developed and performed the analyses and wrote the first version of the manuscript;

    TP developed a preliminary version of the analyses. 
    
    PL and TP provided guidance on the analyses and interpretation of the results and revised the manuscript.
    
    All authors read and approved the manuscript.\\[1cm]
}
%% Les contributions apparaîtrons après
%% \maketitle. Selon les goûts, il est
%% possible de mettre les contributions
%% avant, simplement en les écrivant
%% directement ici. Par exemple :
 % \section*{Contributions de <mon nom> et rôle joué par les coauteurs}
 % J'ai contribué en...
 %
 % Le rôle des coauteurs a été de...
 % \cleardoublepage

%%% INFORMATIONS POUR LA PAGE TITRE
 % Premier auteur·e et adresse
\auteur{Gabriel Dansereau}
% Deuxième auteur·e et adresse (si différente de la première)
\auteur{Pierre Legendre}
%%\adresse{4242 rue imaginaire\\ Universität von der gau\ss sche Zahlenebene}
%%
%% et ainsi de suite pour les autres auteurs
\auteur{Timothée Poisot}
\adresse{Département de sciences biologiques, Université de Montréal\\
  1375 avenue Thérèse-Lavoie-Roux, Montréal, QC, Canada H2V 0B3}
% \adresse{1205 Dr. Penfield Avenue, Montréal, QC, Canada, H3A 1B1 \\ Québec Centre for Biodiversity Science}

\maketitle

\begin{resume}{diversité bêta, unicité écologique, contributions locales à la diversité bêta, modèles de répartition d'espèces, échelle spatiale étendue, eBird}
  
  Objectif: La mesure des contributions locales à la diversité bêta (LCBD) permet d'identifier les sites à forte unicité écologique et ayant une composition en espèces exceptionnelle au sein d'une région d'intérêt. L'utilisation de cette mesure est cependant typiquement restreinte aux échelles locales et régionales avec un petit nombre de sites, puisqu'elle requiert des informations sur la composition complète des communautés parfois difficiles à acquérir à grande échelle spatiale. Dans cette étude, nous examinons comment la mesure des LCBD peut être utilisée pour prédire l'unicité écologique sur de grandes étendues spatiales à l'aide de modèles de répartition d'espèces et de données de science citoyenne.

  Lieu: Amérique du Nord
  
  Période de temps: Années 2000
  
  Taxon étudié: Parulidés
  
  Méthodes: Nous avons utilisé les arbres de régression additifs bayésiens (BARTs) pour prédire la répartition des parulines en Amérique du Nord à l'aide de données provenant de la base de données eBird. Nous avons ensuite calculé les valeurs de LCBD pour les données observées et prédites, puis nous avons examiné la différence par site à l'aide d'une comparaison directe, d'un test d'autocorrélation spatiale et de modèles de régression linéaire généralisés. Nous avons également examiné la variation de la relation entre les valeurs de LCBD et la richesse spécifique entre différentes régions et sur différentes étendues spatiales, ainsi que l'effet de la proportion d'espèces rares sur cette relation.
  
  Résultats: Nos résultats ont montré que la relation entre la richesse et les LCBD varie selon la région et l'étendue spatiale où celle-ci est mesurée et qu'elle est influencée par la proportion d'espèces rares dans la communauté. Les modèles de répartition d'espèces ont fourni des estimés fortement corrélés avec les données observées, mais qui montraient également de l'autocorrélation spatiale.
  
  Principales conclusions: Les sites identifiés comme uniques sur de grandes étendues spatiales peuvent varier en fonction de la richesse régionale, de l'étendue totale et de la proportion d'espèces rares dans la communauté. Les modèles de répartition d'espèces peuvent être utilisées pour prédire l'unicité écologique sur de grandes étendues spatiales, ce qui pourrait permettre d'identifier des points chauds de diversité bêta et d'importantes cibles de conservation au sein de régions non échantillonnées.

\end{resume}

\begin{abstract}{beta diversity, ecological uniqueness, local contributions to beta diversity, species distribution modelling, broad spatial scale, eBird}

  Aim: Local contributions to beta diversity (LCBD) can be used to identify sites with high ecological uniqueness and exceptional species composition within a region of interest. Yet, these indices are typically used on local or regional scales with relatively few sites, as they require information on complete community compositions difficult to acquire on larger scales. Here, we investigate how LCBD indices can be used to predict ecological uniqueness over broad spatial extents using species distribution modelling and citizen science data.
  
  Location: North America.
  
  Time period: 2000s.
  
  Major taxa studied: Parulidae. 
  
  Methods: We used Bayesian additive regression trees (BARTs) to predict warbler species distributions in North America based on observations recorded in the eBird database. We then calculated LCBD indices for observed and predicted data and examined the site-wise difference using direct comparison, a spatial autocorrelation test, and generalized linear regression. We also investigated the relationship between LCBD values and species richness in different regions and at various spatial extents and the effect of the proportion of rare species on the relationship. 
  
  Results: Our results showed that the relationship between richness and LCBD values varies according to the region and the spatial extent at which it is applied. It is also affected by the proportion of rare species in the community. Species distribution models provided highly correlated estimates with observed data, although spatially autocorrelated.
  
  Main conclusions: Sites identified as unique over broad spatial extents may vary according to the regional richness, total extent size, and the proportion of rare species. Species distribution modelling can be used to predict ecological uniqueness over broad spatial extents, which could help identify beta diversity hotspots and important targets for conservation purposes in unsampled locations.

\end{abstract}

% \section{Introduction}
%%
%% Le reste de l'article...
%%

\input{assets/_article1.tex}

% Exemple de citation~: Consultez le \LaTeX\ companien
% de Mittelbach {\it et al\/}\@. \citeyearpar{exemple}.

 % Pour générer la bibliographie à la fin
 % de l'article, utiliser la commande de la
 % classe <dms> \sectionbibliography{<nom du .bib>}.
 % Il y a aussi la possibilité d'utiliser le package
 % <chapterbib>, auquel cas on utilise simplement
 % \bibliography normalement.
 %
 % IMPORTANT : Dans tous les cas, il faut faire
 %    pdflatex these
 %    bibtex chapitre1
 %    bibtex chapitre2
 %    .
 %    .
 %    .
 %    bibtex chapitreN
 %    pdflatex these
 %    pdflatex these
 %
 % où <these> est le nom du .tex principal
 % (qui contient le \documentclass).
 % bibtex a besoin du .aux de chapitre1 et
 % non du .tex. Il est parfois nécessaire
 % d'effacer le .aux et de recommencer la
 % compilation du début.
%%\bibliographystyle{plain} % style plain anglais ou
% \bibliographystyle{plain-fr} % style plain francais
% \bibliographystyle{plainnat-fr}
% \bibliographystyle{apalike-fr}
% \bibliographystyle{apacite}
% \printbibliography
%%\bibliographystyle{<style>} % autre
% \sectionbibliography{references.bib} %Donner le nom du .bib

\endinput
%%
%% End of file `article1.tex'.
