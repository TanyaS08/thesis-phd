%% To change chapter header dynamically from french to english, use
%%\entetedynamique
\setcounter{corA}{0} % Pour recommencer à compter les def,
                     % theo, etc. à partir de 1
\setcounter{secnumdepth}{1}
\renewcommand{\thefigure}{\arabic{figure}}
\setcounter{figure}{0}
 % Pour écrire un article en français
% \francais
 % Pour écrire un article en anglais
\anglais
%% NOTE: La plupart des macros ont un nom en anglais.
%% P.ex. \adresse et \address fonctionnent et sont équivalents.
%% \revue=\journal
%% \auteur=\author
%% \titre=\title

 % Nom de la revue de publication
\revue{Global Ecology and Biogeography}
\article{Evaluating ecological uniqueness over broad spatial extents using species distribution modelling}

 % Contribution(s) personnelle(s) à l'article et rôle joué par tous les coauteurs
 %
 % Nécessaire seulement lorsque vous n'êtes pas seul·e auteur·e.
 % Les contributions peuvent apparaître ailleurs dans la thèse.
 % Si \contributions est laissé vide (p.ex. si vous effacez
 % celui ci-bas), aucune contributions ne seront générées sur
 % la page titre de l'article.
 %
 % REMARQUE : L'exemple est sous forme d'\itemize,
 % mais toutes les constructions \LaTeX\ sont permises.
 % La commande peut aussi contenir un simple petit texte.
 %
 % La commande admet une option [<entête>]
\contributions%[Mes contributions et le rôle des coauteurs]
{
    \begin{itemize}
        \item Developed and performed the analyses;
        \item Wrote the first version of the manuscript;
    \end{itemize}

    Timothée Poisot developed a preliminary version of the analyses. 
    
    Pierre Legendre and Timothée Poisot provided guidance on the analyses and 
    interpretation of the results and revised the manuscript.
    
    All authors read and approved the manuscript.\\[1cm]
}
%% Les contributions apparaîtrons après
%% \maketitle. Selon les goûts, il est
%% possible de mettre les contributions
%% avant, simplement en les écrivant
%% directement ici. Par exemple :
 % \section*{Contributions de <mon nom> et rôle joué par les coauteurs}
 % J'ai contribué en...
 %
 % Le rôle des coauteurs a été de...
 % \cleardoublepage

%%% INFORMATIONS POUR LA PAGE TITRE
 % Premier auteur·e et adresse
\auteur{Gabriel Dansereau}
% Deuxième auteur·e et adresse (si différente de la première)
\auteur{Pierre Legendre}
%%\adresse{4242 rue imaginaire\\ Universität von der gau\ss sche Zahlenebene}
%%
%% et ainsi de suite pour les autres auteurs
\auteur{Timothée Poisot}
\adresse{Département de sciences biologiques, Université de Montréal\\
  1375 avenue Thérèse-Lavoie-Roux, Montréal, QC, Canada H2V 0B3}
% \adresse{1205 Dr. Penfield Avenue, Montréal, QC, Canada, H3A 1B1 \\ Québec Centre for Biodiversity Science}

\maketitle

\begin{resume}{Mots clés}
  Le résumé en français.
\end{resume}

\begin{abstract}{Key words}
  The english abstract.
\end{abstract}

% \section{Introduction}
%%
%% Le reste de l'article...
%%

\input{assets/_article1.tex}

% Exemple de citation~: Consultez le \LaTeX\ companien
% de Mittelbach {\it et al\/}\@. \citeyearpar{exemple}.

 % Pour générer la bibliographie à la fin
 % de l'article, utiliser la commande de la
 % classe <dms> \sectionbibliography{<nom du .bib>}.
 % Il y a aussi la possibilité d'utiliser le package
 % <chapterbib>, auquel cas on utilise simplement
 % \bibliography normalement.
 %
 % IMPORTANT : Dans tous les cas, il faut faire
 %    pdflatex these
 %    bibtex chapitre1
 %    bibtex chapitre2
 %    .
 %    .
 %    .
 %    bibtex chapitreN
 %    pdflatex these
 %    pdflatex these
 %
 % où <these> est le nom du .tex principal
 % (qui contient le \documentclass).
 % bibtex a besoin du .aux de chapitre1 et
 % non du .tex. Il est parfois nécessaire
 % d'effacer le .aux et de recommencer la
 % compilation du début.
%%\bibliographystyle{plain} % style plain anglais ou
% \bibliographystyle{plain-fr} % style plain francais
% \bibliographystyle{plainnat-fr}
% \bibliographystyle{apalike-fr}
% \bibliographystyle{apacite}
% \printbibliography
%%\bibliographystyle{<style>} % autre
% \sectionbibliography{references.bib} %Donner le nom du .bib

\endinput
%%
%% End of file `article1.tex'.
