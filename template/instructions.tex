Fichiers :
 -- README           Ce fichier
 -- dms.cls          La classe
 -- gabaritTPA.tex   Document de travail. Renommer à 
                     un nom convenable (p.ex. titre_nom.tex).
 -- introduction.tex Document de travail contenant l'introduction
 -- article1.tex     Document de travail contenant le premier article.
 -- gabaritTPA.pdf   Document compilé par pdflatex
 -- figures/         Pour placer ses figures

Langues :
  La classe <dms> procure les commandes \francais
  et \anglais pour passer d'une langue à l'autre.
  Elles changent dynamiquement la plupart des mots-clés 
  comme théorème/theorem, lemme/lemma, etc.
  Les entêtes (Chapitre/Chapter, Table des matières/Contents, etc.)
  sont figées une fois que le \begin{document} est passé. Pour 
  qu'elles changent dynamiquement avec \francais et \anglais,
  utilisez \entetedynamique. Attention! Il serait très
  étrange que l'entête d'une chapitre soit Chapitre, mais
  que le suivant soit Chapter. Notez que ceci affecte aussi
  les entrées de la table des matières.

  Le package <babel> n'est donc pas strictement nécessaire,
  mais il reste compatible pour l'utiliser, p.ex.,
  pour la bibliographie ou pour un package comme <natbib>.
  Si <babel> est utilisé, il est tout de même recommandé
  d'utiliser \francais et \anglais plutôt que \selectlanguage,
  car ces premières appellent \selectlanguage lorsque <babel>
  est chargé, mais elles changent quand même les entêtes, ce
  que \selectlanguage ne fera pas.

Thèses par articles :
  IL EST IMPORTANT DE CONSULTER LE GUIDE DE PRÉSENTATION
  DES MÉMOIRES ET DES THÈSES POUR AVOIR L'INFORMATION À
  JOUR.

  En ce jour (2018), une thèse par article doit être en français,
  en moins d'avoir la permission de l'écrire en anglais. Il est
  cependant permis d'inclure ses articles dans d'autres langues. Si un
  article en anglais est inclus, il faut naturellement ajouter
  \anglais au début de l'article.

  La classe offre quelques macros pour faciliter la mise en
  page d'une thèse par article.
  \article{<titre de l'article>}  Débute un article (semblable 
                                  à \chapter ou \section)
  \auteur{<nom de l'auteur>}      Auteur de l'article. L'appeler 
                                  une fois pour chaque auteur
  \adresse{<adresse du            Adresse de l'auteur nommé par
           dernier auteur nommé>} le dernier \auteur
  \begin{resume}[<mots-clés>]     Résumé de l'article en français
  \begin{abstract}[<mots-clés>]   Résumé de l'article en anglais
  \revue[<phrase>]{<nom>}         Revue dans laquelle l'article 
                                  a été publié. La <phrase> est
                                  optionnelle; elle permet d'introduire
                                  la revue autrement que par défaut
                                  (p.ex. « Cette article a été 
                                  soumis à la revue ».)
  \contributions[<phrase>]        (Optionnel) Contributions de 
                   {<texte>}      l'étudiant, s'il n'est pas seul 
                                  auteur (C'est optionel, car les 
                                  contributions peuvent apparaître 
                                  ailleurs, mais elles doivent 
                                  apparaître quelque part.) Il
                                  est important de consulter le
                                  guide. Le rôle de chaque auteur
                                  doit être décrit!
                                  La <phrase> est optionnel; elle
                                  permet d'introduire le <texte>
                                  autrement que la phrase par défaut.
  \maketitle                      Appeler une fois que tous les 
                                  auteurs sont nommés
  \articleenchapitre              \article sera traité comme un 
                                  chapitre plutôt qu'une partie 
                                  (changement estétique, laissé
                                  au choix de l'étudiant)

Options de la classes :
  Au \documentclass, il est possible de passer des options à
  la classe
  
  \documentclass[option1, option2,..., optionN]{dms}
  
  Voici les options disponibles pour la classe <dms> :
  
  phd           Thèse de doctorat (standard ou par articles)
  maitrise       Mémoire de maîtrise
  rapport       Rapport de stage (maîtrise)
  travail       Travail dirigé (maîtrise)
  nobabel       Obselète
  initial       Obselète (Prépare le document au dépôt inital (pages de garde))
  pagetitreart  (Thèse par articles) Place un page titre avant que l'article commence

  Toutes les options habituelles de LaTeX et de la classe 
  <amsbook> sont aussi disponibles. En voici quelques unes 
  utiles ou recommandées :

  12pt     Police de caractères en 12pt (10pt par défaut)
  twoside  Prépare le document pour une impression recto-verso
  reqno    Numéro d'équation à droite (par défaut pour la classe <dms>)
  leqno    Numéro d'équation à gauche
  draft    Montre les défauts de la mise en page et
           empêche l'inclusion d'images (utile pour
           un document en construction lourd à compiler 
           à cause d'un grand nombre de figures)


Document scindé :
  Pour des documents d'envergure, comme une thèse, il est
  recommandé de diviser le ficher .tex principal en petits
  fichiers <chapitre*.tex>.  Ensuite, ce chapitre est
  inclus dans le fichier principal par
  
  \include{chapitre1}  %chapitre*
  
  Le fichier chapitre1.tex ne doit PAS contenir ni préambule
  (\documentclass, \usepackage, etc.), ni \begin{document},
  ni \end{document}.  Par conséquent, chapitre1.tex ne peut
  pas être compilé seul.

Bibliographie par article :
  Pour que chaque article ait sa propre bibliographie,
  les articles doivent être dans leur propre fichier .tex
  (p.ex. article1.tex, article2.tex, etc.)
  
  La classe met à la disposition de l'utilisateur les
  deux macros suivantes 

  \sectionbibliography{<fichier .bib>} % La biblio sera une section
  
  \chapterbibliography{<fichier .bib>} % La biblio sera un chapitre

  Il ne faut pas utiler \bibliography comme un document normal. 
  Il faut bien sûr choisir le style. Par exemple, le style plain
  s'appelle par 

  \bibliographystyle{plainfr}

  pour un article en français.
  La compilation se fait comme suit

  pdflatex gabaritTPA
  bibtex article1
  bibtex article2
  .
  .
  .
  bibtex articleN
  pdflatex gabaritTPA
  pdflatex gabaritTPA

  Comme note finale, il existe aussi le package <chapterbib> qui
  fait le même travail. Vous pouvez vous servir de ce package
  au lieu de \sectionbibliography si vous le désirez.
